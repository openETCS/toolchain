VERSION: 20130801\_openETCS\_ICA\_v01.docx

openETCS INDIVIDUAL COMMITTER AGREEMENT (oE-ICA)

THIS INDIVIDUAL COMMITTER AGREEMENT (THE “AGREEMENT”) is entered into as of
the \_\_\_\_\_\_\_\_\_\_\_\_\_\_\_\_\_\_\_\_ (the “Effective Date”) by and between the openETCS Project
Consortium consisting of organizations as listed on the ITEA2 website, ref:
http://www.itea2.org/project/ index/view/?project=10135 , AND having signed the openETCS
Project Cooperation Agreement (“oE-PCA”) or openETCS Declaration of Acceptance (“oE-
DoA”), represented by the openETCS Project-Coordinator
and \_\_\_\_\_\_\_\_\_\_\_\_\_\_\_\_\_\_\_\_ (“Committer”) an individual listed in the openETCS Committer

Database who has been approved to be a committer as further described herein.

INTRODUCTION

Individuals, who give frequent and valuable contributions to the openETCS development
project, or component of the project, can have their status promoted to that of a "committer"
for that project or component respectively, in accordance with the project’s corresponding
charter. A committer has write access to the source code repository for the associated
project (or component), or to other content on the openETCS Project website. In order for an
individual to become a committer, another committer for the project (or component) must
nominate that individual. Once an individual is nominated, the existing committers for the
project (or component) will vote using the process and rules determined by the Project
Charter and administered by the Project Management Committee (“PMC”). When a new
project is started, the responsible corresponding Development Team Leader PMC (Project
Management Committee) will nominate an initial set of committers for approval by the
openETCS PCC Chairman (or his delegates). Becoming a committer is a privilege that is
earned by contributing and showing discipline and good judgment. It is a responsibility that
should be neither given nor taken lightly.

By executing this Agreement, Committer agrees that: (a) he or she has reviewed this
Agreement, (b) he or she shall comply with all obligations that result from being a committer,
(c) he or she shall be entitled to enjoy the rights of a committer, subject to the terms hereof
and (d) he or she is fully entitled to license his or her contributions under the terms of the
EUPL and/or the Creative Commons by-sa 3.0 unported license (“cc-by-sa”) respectively
(see 1.3 below).

1. COMMITTER RIGHTS AND OBLIGATIONS

1.1 Compliance with Bylaws. Committer agrees to abide by the Bylaws of the openETCS
Project Consortium as may be amended from time to time, which is hereby incorporated
herein by reference.

1.2 Compliance with Policies and Guidelines. Committer agrees to abide by the IP Policy
and Committer Guidelines and any and all additional policies, guidelines and procedures
adopted by the openETCS Project Consortium, as may be amended from time to time, which
are hereby incorporated herein by reference.

1.3 Compliance with the European Public License, the Creative Commons by-sa 3.0
unported license and the openETCS Terms of Use. All contributions of Committer
submitted to the openETCS repository will be published under the terms of the European
Public License (“EUPL”) and (concerning Non-code contributions) under the terms of the
EUPL and the Creative Commons by-sa 3.0 unported license (“cc-by-sa”). Committer
represents that he or she has reviewed and understands the terms and conditions of the
EUPL, the cc-by-sa and the openETCS Terms of Use currently located at:
http://openetcs.org/termsofuse/. Committer agrees that the openETCS Terms of Use will
serve as the general contribution agreement for the openETCS project, unless otherwise
agreed to in accordance with the Bylaws and IP Policy. Except as otherwise determined by
the openETCS PCC in accordance with the Bylaws and IP Policy, or as set forth in the
openETCS Terms of Use, Committer agrees that the EUPL will serve as the distribution
license for software contributions and the EUPL and cc-by-sa as the distribution licenses for
Non-code contributions (dual-licensing).

1.4 Committer Questionnaire. Committers shall complete and submit to the openETCS
Committer Questionnaire.

1.5 openETCS Committer Employer Consent Form. If Committer is employed or is
otherwise performing services for a third party as an independent contractor, Committer shall
have an authorized representative of such employer sign and submit directly to the
openETCS project office the Committer Employer Consent Form provided by the openETCS
project office.

1.6 Committer Contact Information. Committers shall promptly inform the openETCS
project office of any change in the information provided on the Committer Questionnaire,
including without limitation address, other contact information and any change in employment
or independent contractor status, as well as the change in employer or third party to whom
services are being provided. Committer shall, upon the openETCS project offices’s request,
confirm the currency of all information provided in the Committer Questionnaire.

1.7 Committer’s rights. Committers working on content in the openETCS.org Github
repository may be granted commit rights to specific project directories and/or files in the
repository. Committers working on content on the openETCS.org web site may be granted
WebDAV access to specific web site directories and/or files. Committers may also be
granted other rights necessary to administrate and manage projects such as mailing list
administration, Bugzilla administration, etc. The openETCS project office will have complete
control and discretion over which capabilities are assigned to a Committer account, and may
terminate or temporarily disable Committer access for any reason at any time.

1.8 Treatment of Account. Each Committer shall maintain the strict confidentiality of his or
her passwords issued by the openETCS project office (“Password”) and shall not allow any
other individual or entity to use his or her username or Password. Should a Committer
become aware of any such use, Committer shall notify the openETCS project office
immediately by sending an e-mail to helpdesk@openETCS.org, or such other e-mail address
as may be designated by the openETCS project from time to time. Until a Committer has
provided such notice to the openETCS project, such Committer shall be presumed to have
taken and shall be fully responsible for all actions made through its username and Password.

2. TERM AND TERMINATION

2.1 Term. The term of this Agreement shall begin on the Effective Date and shall continue
until terminated by either party, with or without cause, by written notice to the other party.

3. GENERAL

3.1 No Other Licenses. By executing this Agreement, Committer neither grants nor receives,
by implication, estoppel, or otherwise, any rights under any copyright, patents or other
intellectual property rights of the openETCS project or any Member.

3.2 Limitation of Liability. IT IS THE EXPECTATION OF THE OPENETCS PROJECT
CONSORTIUM AND ITS MEMBERS THAT COMMITTER WILL MEET COMMITTER’S
OBLIGATIONS, AND NOT EXCEED THE SCOPE OF HIS OR HER AUTHORITY, AS SET
FORTH IN THIS AGREEMENT. NOTWITHSTANDING THE PRECEDING SENTENCE, IN
NO EVENT WILL EITHER OPENETCS PROJECT CONSORTIUM, ITS MEMBERS OR
COMMITTER BE LIABLE TO EACH OTHER OR ANY MEMBER OR THIRD PARTY UNDER
THIS AGREEMENT FOR THE COST OF PROCURING SUBSTITUTE GOODS OR
SERVICES, LOST PROFITS, LOST REVENUE, LOST SALES, LOSS OF USE, LOSS OF
DATA OR ANY DIRECT, INCIDENTAL, CONSEQUENTIAL, INDIRECT, PUNITIVE, OR
SPECIAL DAMAGES WHETHER OR NOT SUCH PARTY HAD ADVANCE NOTICE OF
THE POSSIBILITY OF SUCH LOSSES OR DAMAGES.

3.3 Governing Law. This Agreement shall be construed and controlled by the laws of
Germany without reference to conflict of laws principles.

3.4 Notices. All notices or other communications to or upon any party shall be delivered to or
at the addresses set forth on the signature page(s) hereto. For purposes of this Section,
notice can include notice by written mail, electronic mail or by facsimile and shall be deemed
served when sent; provided, however, that notice of a breach of this Agreement and notice of
termination of this Agreement shall be given by overnight courier service or certified mail,
return receipt requested. Either party may give written notice of a change of address and,
after notice of such change has been received, any notice or request shall thereafter be
given to such party at such changed address.

3.5 Complete Agreement; No Waiver. Except with respect to the Bylaws of openETCS
Project Consortium, the IP Policy, the EUPL, the cc-by-sa (if applicable), the openETCS.org
Terms of Use and any other policies, guidelines and procedures that may be adopted by
openETCS Project, from time to time, in accordance with the Bylaws, this Agreement,
including all attachments, sets forth the entire understanding of openETCS project, and the
Committer with respect to the subject matter hereof and supersedes all prior agreements and
understandings relating hereto, unless otherwise stated in this Agreement. The waiver of any
breach or default will not constitute a waiver of any other right hereunder or any subsequent
breach or default.

3.6 Counterparts. This Agreement may be executed in one or more counterparts, each of
which shall be deemed an original, but collectively shall constitute one and the same
instrument.

3.7 Compliance with Laws. Anything contained in this Agreement to the contrary
notwithstanding, the obligations of openETCS project and Committer shall be subject to all
laws, present and future, of any government having jurisdiction over openETCS project or
Committer including, without limitation, all export and re-export laws and regulations. It is the
intention of openETCS project and Committer that this Agreement and all referenced
documents shall comply with all applicable laws and regulations.

3.8 Independent Contractors. The relationship of openETCS project with respect to
Committer established by this Agreement is that of independent contractors. This Agreement
does not give either party the power to direct and control the day to day activities of the other,
constitute the parties as partners, joint venturer, co-owners, principal-agent or otherwise
participants in a joint or common undertaking, or, except as expressly provided herein, allow
either party to create or assume any obligation on behalf of the other for any purpose
whatsoever.

