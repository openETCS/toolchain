 \documentclass{openetcs_report}
% Use the option "nocc" if the document is not licensed under Creative Commons
%\documentclass[nocc]{template/openetcs_article}
\usepackage{todonotes}
\usepackage{appendix}
\usepackage{lipsum,url}
\usepackage{pdfpages}
%\usepackage{bibtopic} % Multibib
\usepackage{booktabs}
\usepackage{hyperref}

\usepackage[section,                 % add the glossary to the table of content 
            description,             % acronyms have a user-supplied description,
            style=superheaderborder, % table style
            nonumberlist             % no page number
]{glossaries}
\hypersetup{
linkbordercolor 	={1 1 1}}

%===========================
% Graphicpath
%===========================
\graphicspath{{./template/}{.}{./images/}}

%===========================
% Abbreviation file
%===========================
% \renewcommand*{\glossaryname}{List of Terms}
% \makeglossaries
% \loadglsentries{wp7_glossary} 
%===========================
%===========================
% Todo note margin
%===========================
\setlength{\marginparwidth}{7em}
\let\oldmarginpar\marginpar
\renewcommand\marginpar[1]{\-\oldmarginpar[\raggedleft\footnotesize #1]%
{\raggedright\footnotesize #1}}
%===========================

\begin{document}

\frontmatter
\project{openETCS}

%Please do not change anything above this line
%============================
% The document metadata is defined below

%assign a report number here
\reportnum{OETCS/WP7/D7.3}

%define your workpackage here
\wp{Work-Package 7: ``Toolchain''}

%set a title here
\title{Toolchain Qualification Process Description}

%set a subtitle here
%\subtitle{}

%set the date of the report here
\date{January 2014}

%define a list of authors and their affiliation here

\author{Cecile Braunstein \and Jan Peleska}

\affiliation{University Bremen}



% define the coverart
\coverart[width=350pt]{openETCS_EUPL}

%define the type of report
\reporttype{Qualification process description}


\begin{abstract}
%define an abstract here
This document presents  different ideas of a toolchain
qualification. It describes a process  for the openETCS toolchain
qualification.  
\end{abstract}


%=============================
%Do not change the next three lines
\maketitle
\tableofcontents
%\listoffiguresandtables

\newpage
%=============================

% The actual document starts below this line
%=============================
%Start here
%=============================
% Document Managment
%=============================
\chapter{Document Information}
\marginpar{This document has too many elipses (...).  Each instance should be removed and clarified.}

\begin{tabular}{|p{4.4cm}|p{8.7cm}|}
\hline
\multicolumn{2}{|c|}{Document information} \\
\hline
Work Package &  WP7  \\
Deliverable ID or doc. ref. & D7.3\\
\hline
Document title & Toolchain Qualification Process Description \\
Document version & 01.00 \\
Document authors (org.)  & Cécile Braunstein and Jan Peleska (Uni.Bremen) \\
\hline
\end{tabular}

\begin{tabular}{|p{4.4cm}|p{8.7cm}|}
\hline
\multicolumn{2}{|c|}{Review information} \\
\hline
Last version reviewed &  \\
\hline
Main reviewers &  \\
\hline
\end{tabular}

\begin{tabular}{|p{2.2cm}|p{4cm}|p{4cm}|p{2cm}|}
\hline
\multicolumn{4}{|c|}{Approbation} \\
\hline
  &  Name & Role & Date   \\
\hline  
Written by    &  Cécile Braunstein & WP7-T7.3 Sub-Task  & 12.01.2014 \\
&  & Leader&\\
\hline
Approved by &  &   &  \\
\hline
\end{tabular}

\begin{tabular}{|p{2.2cm}|p{2cm}|p{3cm}|p{5cm}|}
\hline
\multicolumn{4}{|c|}{Document evolution} \\
\hline
Version &  Date & Author(s) & Justification  \\
\hline  
00.00 & 12.01.2014 & C. Braunstein  &  Document creation  \\
01.00 & 19.01.2014 & C. Braunstein  &  Jan Peleska suggestions \\
01.01 & 28.01.2014 & M. Jastram  &  Review \\


\hline  
\end{tabular}
\newpage
%==========================================


%------ List of terms and definition ----------------
%\printglossary
%==========================================
\mainmatter
%----------------------







\chapter{Introduction to Toolchain Qualification}
\label{chap-1}
\section{Tool Qualification}
\label{sec-1-1}


The CENELEC EN 50128 standard \cite{standard_railway_2011} defines the tool
qualification as follows:\\
{\it ``The objective is to provide evidence that potential
failures of tools do not adversely affect the integrated tool-set output in a
safety related manner that is undetected by technical and/or organizational
measures outside the tool. To this end, software tools are categorized into
three classes namely, T1, T2 \& T3 respectively.''}

We recall here the different class definitions:
\begin{itemize}
\item Tool class T1: No generated output can be used directly or indirectly to the
  executable code;
\item Tool class T2: Verification tools, the tool may fail to detect errors or
  defects;
\item Tool class T3: Generated output directly or indirectly  as part of the
  executable code.
\end{itemize}
The deliverable D2.2 \cite{pokam_report_2013} summarizes the requirements for the
tool needed by the different tool classes. The report highlights that the
effort differs depending on the tool class. Furthermore, for
the most critical class T3,  the evidence should be provided that the output is
conform to the specification or that \emph{any failure in the output
  are detected}. 

The standard defined how to classify each tool individually (see
\cite{nielsen_efficient_2012,huang_test_2013} as an example).  But dealing with a tool
chain, integrated within a tool platform, implies extra effort to
ensure that the tool integration does not introduce new errors. For
example mechanism such as artifacts versioning, time-stamping
operations, etc ... should also be considered when qualifying the tool
chain.

In summary, the effort of qualification depends on the tool class and the tool
error detection capabilities. To reduce the cost of the toolchain qualification
and regarding the fact that our development imply regular releases, a systematic
toolchain analysis approach has to be defined.
 
\section{Toolchain Qualification State of the Srt}
\label{sec-1-2}
Some recent works have been done in the field of toolchain
qualification from a variety of projects. The next section summarizes
the most significant ones.

\subsection{Slotosh and al. (project RECOMP)}
\label{sec-1-2.1}

 \cite{slotosch_model-based_2012} describes a model-based approach to tool
 qualification to comply with DO-330 and integration into the Eclipse
 development environment. The authors claim that the benefits of their
 method  are the following:

\begin{description}
\item[Clarity:] remove ambiguities;
\item[Re-usability and Transparency:] check for reuse in different toolchain;
\item[Completeness:] the model covers all parts of the  development
  process and tis traceable;
\item[Automation:] Some part of the process may be automated.
\end{description}

Their method is explained in detail in \cite{slotosch_iso_2012}: the
toolchain analysis is based on a domain specific toolchain model
they have defined. This model is used to represent the toolchain
structure as well as the tool confidence.  Their goal is to deduce the
tool confidence level and to expose specific qualification
requirements, furthermore, their idea is not only to check tool by
tool but to have a more holistic approach and makes use of
rearrangement and/or the extension of the toolchain to avoid the
certification of all tools. This allows them to reduce the
qualification effort by focusing only on the critical tools and make
use of already available information.  Moreover, some inconsistencies
check such as, missing description, unused artifacts ..., may be
automatically done by using the toolchain model. Finally, they
provide support to automatize the document production.

\cite{wildmoser_determining_2012} apply their tool and methods an
industrial use case to determine the potential errors in the tool-chain.

\subsection{Asplund and al. (projects iFEST, MBAT)}
\label{sec-1-2.2}

The authors investigate the question if there exists part of the environment related to tool
integration that may fall outside the tool qualification defined by the a norm
(ISO 26262 here \cite{asplund_qualifying_2012}). And if so, how tool integration
is affected by ensuring functional safety. One conclusion is that the tool
integration may lead to increase the qualification effort.

They also state that the standards (EN 50128, DO-178C and ISO26262)
are not sufficient to check safety of a toolchain, but some part of a
toolchain may be taken into account to mitigate the qualification
effort. 
They highlight 9 safety issues caused by tool integration that also
allow to be more exact when identifying software that have to be
qualified for certification purpose. 

They advocate that to deal with the qualification of tool integration
within a toolchain a system approach should be taken, we should not
thing about individual tools anymore.  Their proposed method is a ``System
Approach'' for toolchain qualification following these steps:

\begin{enumerate}
\item Pre-Qualification of development tool (requirements tools,design
  tools ...): provided by the vendors.
\item Pre-qualification at the tool-chain level: based on step 1 and
  reference work-flows, defined where are the safety critical part.
\item Qualification of the tool-chain: check differences of step 2 and
  the actual deployed toolchain.
\item Qualification at the tool level: based on the actual environment
  when deploying the toolchain.
\end{enumerate}
This approach leads them to separate the parts required to software tool
qualification and to identify safety issues related to tool integration.

In \cite{asplund_towards_2012}, they explore the step 2): identifying
the required safety goals due to tool integration and obtain a
description of a reference work-flow and tool-chain with annotation
about the mitigating effort.  They proposes to use the TIL language, a
domain specific language for toolchain model.  The model of the tool
chain is used to perform a risk analysis and to annotate parts
that need mitigating effort for the safety issues due to tool
integration. 

\subsection{Biehl and al. (projects CESAR, iFEST, MBAT)}
\label{sec-1-2.3}

Biehl proposed a Domain Specific Language named TIL for Generating Tool Integration
Solutions \cite{biehl_matthias_domain_2011}.  A toolchain is described in terms
of a number of ``Tool Adapters'' and the relation between them.
\begin{itemize}
\item ToolAdapters: exposes data and functionality of a tool
\item Channels
  \begin{itemize}
  \item ControlChannels describe service calls
  \item DataChannel describe data exchanges
  \item TraceChannel  describe creation of a trace links
  \end{itemize}

\item Sequencer: describe sequential control flow (sequence of services)
\item User:  describe and limit the possible interaction
\item Repository:  provide storage and version Management of tool data
\end{itemize}
This DSL allows early analysis of the toolchain.
It may generate part of tool adapter code based on the source and target
meta-model.

More recently, Biehl and al.  define a standard language for modeling
development process defines by OMG 2008. The language has been used in
\cite{biehl_constructing_2012,biehl_early_2012} together with the TIL
language to tailor a toolchain following a process model. The goal is
to be able to model both the development process and the set of tools
used.  A process is defined as follows:
\begin{itemize}
\item Process: several Activities
\item Activity: set of linked Tasks, WorkProducts, Roles
\item A Role can perform a Task
\item A WorkProduct can be anaged by a Tool
\item A Task can use a Tool
\end{itemize}
Using together the process development language  and the toolchain
language, in \cite{biehl_early_2012}, the authors  measure the alignment of a toolchain
with a product development process. The method proceeds as follows:
\begin{enumerate}
\item Inputs:
  \begin{itemize}
  \item formalized description of the toolchain design
  \item description of the process including the set of tools and their capabilities
  \end{itemize}
\item Initial verification graph
\item Automatic mapping links to the verification graph (acc. to mapping rules)
\item Apply alignment rule on the verification graph
\item Apply metrics to determine the degree of alignment btw the tool-chain and the
   process
\end{enumerate}
The metrics and the misalignment list provide feedback to refine the tool-chain
design.



\chapter{OpenETCS Toolchain Qualification Process}
\label{chap-2}

\section{Toolchain Analysis}
\label{sec-2-1}

All the methods mentioned above start with a complete definition of the tool
chain. In OpenETCS, the development of the toolchain follow an \emph{agile} method,
hence for each (major) release we have to deal with an incomplete tool
chain. In addition to the methods of the previous section,  we need a qualification
process that can adapt to the development speed, deal with incomplete toolchain
and can re-use qualification information.

Moreover, as stated by Asplund and al., the toolchain provides some mechanism
that has to be also ensured, reducing thereby the effort for each tool. These
safety goals are related to the tool integration. In our context, most of  the tool
integration is made by integrated tools into a tool platform.
From the previous cited paper, the tool platform should ensure the following
safety-goals
\begin{itemize}
\item Coherent Time Stamp Information: common time stamps on development artifacts.
\item Notification: the user should be notified when artifacts changed.
\item Data integrity:  avoid use of obsolete artifacts, the data used reflects the
  current state.
\item Data Mining: all data used by safety analysis should be available and be
  verifiable.
\end{itemize}

\section{OpenETCS Toolchain Qualification Process}
\label{sec-2-2}
\marginpar{I don't understand this section.}

The OpenETCS toolchain is described as a SysML activity diagram\marginpar{provide link.}. This
activity diagram grows according to the new feature request and the
need of openETCS participants.  \marginpar{The diagram may be outdated - the ``truth'' is found in the Eclipse Product Definition.}
Each feature of the toolchain is
represented as an activity node, each artifacts by a data store.  Each
feature realizes at least one use case and is implemented by at least
one tool. Note that in the tool platform environment tool may also be
implemented as plug-ins\marginpar{It is not clear what a ``tool'' is (vs. feature, plug-in or toolchain)}
.  The diagram also represents the order
between the different features, it also shows which tasks maybe done
in parallel and which ones are dependent of other tasks.

To mitigate the qualification process, we will consider each feature and not
each tool  since the combination of tool may, for example, ensure the error
detection capability of a feature.\marginpar{???}  Furthermore, the toolchain is a collection
of feature and not tools, this differs from Asplund and al. in the sense that
some of the tool integration mechanism Automated Transformation of Data are part
of the feature and are not falling out of the scope of the qualification.

Due to our development process, a ``pre-qualification'' of tools should be made
when integrating a tool.

\marginpar{This high-level list should be detailed.}
\paragraph{Tool Integration Process for Qualification}

\begin{itemize}
\item Define name and version
\item Describe use cases
\item Provide input/output artifacts format (associated with the
  version)
\item Integrate the tool in the SysML model
\item Provide tool manual and other available documentation (associated with the version)
\item Link with an issue tracker
\end{itemize}

One possible implementation is to represent all these in formations
directly in the SysML model.

\marginpar{This high-level list should be detailed.  What I would expect: Which artefacts exist; How they are connected; When artefacts have to be re-validated (e.g. due to changes); What roles exist; who is responsible for what; etc.}
\paragraph{The Qualification Process}

\begin{enumerate}
\item Feature Analysis
  \begin{itemize}
  \item This step should assign a class to each feature based on the use cases.
  \item Define the potential errors
  \item Identify counter measure and/or error detection
  \item For T3 tools 2 alternatives:  certified compiler/generator or
    object code checker and/or exhaustive tester
  \end{itemize}
\item Tool platform  analysis 
  \begin{itemize}
  \item Provide evidence of the safety-goals mentioned in the
    previous sub-section
  \end{itemize}
\item Toolchain Analysis
  \begin{itemize}
  \item Defines the work-flow
  \item Identify the ``hot spots'' of the toolchain
  \item Rearrange the toolchain if possible
  \item Find new measures when needed with combining tools (redundancy with orthogonal
    codes \ldots{})
  \end{itemize}
\item Toolchain qualification verification 
  \begin{itemize}
  \item check consistency of tool version with  manuals \ldots{}
  \item Generate table to  check if all possible errors has a
    detection or a correction mechanism
\item Generate the qualification report
  \end{itemize}

\end{enumerate}

\bibliographystyle{plain}
\bibliography{D7.4}
\end{document}