\chapter{SysML with Entreprise Architect}

\begin{description}
\item[\textcolor{green}{Author}] Author of the approaches description  Cécile Braunstein (Uni. Bremen)
\item[\textcolor{blue}{Assessor 1}] First assessor of the approaches Uwe Steinke (Siemens)
\item[\textcolor{magenta}{Assessor 2}] Second assessor of the approaches Roberto Kretschmer (TWT)
\end{description}

In the sequel, main text is under the responsibilities of the author.

\begin{author_comment}
Author can add comments using this format at any place.
\end{author_comment}

\begin{assessor1}
First assessor can add comments using this format at any place.
\end{assessor1}

\begin{assessor2}
Second assessor can add comments using this format at any place.
\end{assessor2}

When a note is required, please follow this list :
\begin{description}
\item[0] not recommended, not adapted, rejected
\item[1] weakly recommended, adapted after major improvements, weakly rejected
\item[2] recommended, adapted (with light improvements if necessary)  weakly accepted
\item[3] highly recommended, well adapted,strongly accepted
\item[*] difficult to evaluate with a note (please add a comment under the table)
\end{description}

All the notes can be commented under each table.

\section{Presentation}

This section gives a quick presentation of the approach and the tool.

\begin{description}
\item[Name] SysML modelisation with Enterprise Architect
\item[Web site] \url{http://www.sparxsystems.com.au/}
\item[Licence] Commercial
\end{description}

\paragraph{Abstract} Short abstract on the approach and tool 
SysML \cite{SysML} is a graphical language that extends UML for a customize
version suitable for system engineering. It may help modeling system within a board range of
system variety that may include hardware, software, data, personnel and facilities. It supports the
specification, analysis design, verification and validation of complex system.

Enterprise architect (EA) version 9.3 \cite{website} has been used to implement
the model. EA provides SysML and UML modeiling capabilities.
EA is a visual platform for designing and constructing
software systems, for business process modeling, and for more generalized modeling purposes. it
covers all aspects of the development cycle. The main advantages is the requirement management
and tracing, the team work and the include versionning. The main cons : it is not an open source
tool.



\paragraph{Publications} Short list of publications on the approach 
\begingroup
\renewcommand{\chapter}[2]{}%
\renewcommand{\bibname}{}
\begin{thebibliography}{5}

\bibitem{SysML}Object Management Group,  {\it OMG Systems Modeling Language (OMG
SysML\textsuperscript{TM})}, \url{www.omgsysml.org}, 2012.
\bibitem{website}Some useful links are given at \url{http://www.sparxsystems.com.au/products/ea/trial.html}.
\bibitem{sysmlbook} Jon Holt and Simon Perry,\it{SysML for Systems
    Engineering},IET, 2008.
 
\end{thebibliography}
\endgroup



\section{Main usage of the approach}
\label{main_usage}
This section discusses the main usage of the approach.

According to the figure \ref{fig:main_process}, for which phases do you recommend the approach (give a note from 0 to  3) :

\begin{tabular}{|l | c | c | c | c|}
\hline
& \textcolor{green}{Author} & \textcolor{blue}{Assessor 1} & \textcolor{magenta}{Assessor 2} & Total \\
\hline 
System Analysis & 3  & & &  \\
\hline
Sub-system formal design & 3 & & & \\
\hline
Software design & 3 & & & \\
\hline
Software code generation & 1 & & & \\
\hline
\end{tabular}
\begin{author_comment}
Enterprise architect is able to generate C, java ... code from the UML/SysML
representation. This capability has not been tested by me. Moreover
from the serialized form of the model (xmi) it is possible to use
other tool to perform different task including code gneration.
\end{author_comment}
According to the figure \ref{fig:main_process}, for which type of activities do you recommend the approach (give a note from 0 to  3) :

\begin{tabular}{|l | c | c | c | c|}
\hline
& \textcolor{green}{Author} & \textcolor{blue}{Assessor 1} & \textcolor{magenta}{Assessor 2} & Total \\
\hline 
Documentation &2 & & &  \\
\hline
Modeling & 3& & &  \\
\hline
Design &3 & & & \\
\hline
Code generation &2 & & & \\
\hline
Verification & 1 & & & \\
\hline
Validation & 1 & & & \\
\hline
Safety analyses & 2  & & & \\
\hline
\end{tabular}
\begin{author_comment}
As said before, the model from enterprise architect may be imported
with a bit of effort (adapt parser to EA xmi) into other tools or some
tool may be plugged to EA. The model may be then used for the
different tasks of the table.
\end{author_comment}


\paragraph{Known usages} Have you some examples of usage of this
approach to  compare with the OpenETCS objectives ?

Plenty of experiences using SysML or UML in embedded system may be
found, the reader may refer to the proceeding of international
conference such as : Conference on Systems Engineering
Research (CSER), International Conference on Model-Based Systems Engineering
(MBSE), Embedded realtime software and systems (ERTS).
 

\section{Language}
This section discusses the main element of the language.

Which are the main characteristics of the language :

\begin{tabular}{|l | c | c | c | c|}
\hline
& \textcolor{green}{Author} & \textcolor{blue}{Assessor 1} & \textcolor{magenta}{Assessor 2} & Total \\
\hline 
Informal language & 0 & & &  \\
\hline 
Semi-formal language & 2 & & &  \\
\hline
Formal language & 3 & & &  \\
\hline
Structured language &3 & & & \\
\hline
Modular language &3 & & & \\
\hline
Textual language & 1 & & & \\
\hline
Mathematical symbols or code &1 &  & & \\
\hline
Graphical language &3 & & & \\
\hline
\end{tabular}

According WP2 requirements, give a note for the capabilities of the language (from 0 to 3) :

\begin{tabular}{|l | c | c | c | c|}
\hline
& \textcolor{green}{Author} & \textcolor{blue}{Assessor 1} & \textcolor{magenta}{Assessor 2} & Total \\
\hline
Declarative formalization of properties (D.2.6-X-28) &2 & & & \\
\hline
Simple formalization of properties (D.2.6-X-28.1) &2 & & & \\
\hline
Scalability : capability to design large model &3 & & & \\
\hline
Easily translatable to other languages (D.2.6-X-30) &2 & & & \\
\hline
Executable directly (D.2.6-X-33) &1 & & & \\
\hline
Executable after translation to a code (D.2.6-X-33) & 2 & & & \\
(precise if the translation is automatic) &2 & & & \\
\hline
Simulation, animation (D.2.6-X-33) &1 & & & \\
\hline
Easily understandable (D.2.6-X-27) &3 & & & \\
\hline
Expertise level needed (0 High level, 3 few level) &2 & & & \\
\hline
Standardization (D.2.6-X-29) &3 & & & \\
\hline
Documented (D.2.6-X-29) &3 & & & \\
\hline
Extensible language (D.2.6-01-28) &3 & & & \\
\hline
\end{tabular}


\paragraph{Documentation} 
SysML is  defined and standardized by OMG. They provide a complete
definition of the language and a guideline on the methodology
\cite{SysML}. The serialized form (the textual representation of the
graphical model) is also defined and standardized by OMG (see the XMI
definition).

Enterprise architect gives tutorial and sample model to start
with. The tool is really easy to use.

\paragraph{Language usage} 
SysML is dedicated on system engineering and suitable for a board range of
system variety. Using stereotype and profile the language may be
extended and customized according to the user need following the
specific domain of the modeled system.


\section{System Analysis}
This section discusses the usage of the approach for system analysis.
It can be skipped depending the results of \ref{main_usage}.

Acoording WP2 requirements, how the approach can be involved for the sub-system requirement specification ?

\begin{tabular}{|l | c | c | c | c|}
\hline
& \textcolor{green}{Author} & \textcolor{blue}{Assessor 1} & \textcolor{magenta}{Assessor 2} & Total \\
\hline
Independent System functions definition (D.2.6-X-10.2.1)  &3 & & &  \\
\hline 
System architecture design (D.2.6-X-10.2) &3 & & &  \\
\hline
System data flow identification (D.2.6-X-10.2.3)  &3 & & &  \\
\hline
Sub-system focus (D.2.6-X-10.2.4)  &3 & & &  \\
\hline
System interfaces definition (D.2.6-X-10.2.5)  &3 & & &  \\
\hline
System requirement allocation (D.2.6-X-10.3)  &3 & & &  \\
\hline
Traceability with SRS (D.2.6-X-10.5)  &3 & & &  \\
\hline
Traceability with Safety activities (D.2.6-X-11)  &3 & & &  \\
\hline
\end{tabular}
\begin{author_comment}
  SysML proposes a choice of diagrams to describes the system or the
  software under consideration. It is then possible to represent the
  system at different level of abstraction. Moreover one of the main
  addition of SysML compare to UML is the requirement diagram that
  makes the link between the model and the requirements.
\end{author_comment}


\section{Sub-System formal design}
This section discusses the usage of the approach for sub-system formal design.
It can be skipped depending the results of \ref{main_usage}.

Two kinds of model can be planned during this phase: semi-formal models to  cover the SSRS (D.2.6-X-12.1) and strictly formal  models to  focuss on some functional and safety aspects (D.2.6-X-14).  Obviously some strictly  formal means can be used to define the semi-formal  model.

\subsection{Semi-formal model}

Concerning semi-formal model, how the WP2 requirements are covered ?

\begin{tabular}{|l | c | c | c | c|}
\hline
& \textcolor{green}{Author} & \textcolor{blue}{Assessor 1} & \textcolor{magenta}{Assessor 2} & Total \\
\hline 
Consistency to SSRS (D.2.6-X-12.2) &3 & & &  \\
\hline
Coverage of SSRS (D.2.6-X-12.2.1)  &3 & & &  \\
\hline
Coverage of SSHA (D.2.6-X-12.2.2)  & * & & &  \\
\hline
Management of requirement justification (D.2.6-X-12.2.3)  &3 & & &  \\
\hline
Traceability to  SSRS (D.2.6-X-12.2.5)  & 3 & & &  \\
\hline
Traceability of exported requirements (D.2.6-X-12.2.6)  &3 & & &  \\
\hline
Simulation or animation (D.2.6-X-13 partial)  &2 & & &  \\
\hline
Execution (D.2.6-X-13 partial)  &2 & & &  \\
\hline
Extensible to strictly formal model (D.2.6-X-14.3) &3 & & &  \\
\hline
Easy to  refine towards strictly formal model (D.2.6-X-14.4) &3 & & &  \\
\hline
Extensible and modular design (D.2.6-X-15)  &3 & & &  \\
\hline
Extensible to software architecture and design (D.2.6-X-30)   &3 & & &  \\
\hline
\end{tabular}
\begin{author_comment}
The link with the SSRS may be really easy if the SSRS is modeled
with SysML.

For the coverage, I do not know and I did not try.


Simulation,animation  and execution may be done by additional plug-in
or by different tools.
\end{author_comment}
Concerning safety properties management, how the WP2 requirements are covered ?

\begin{tabular}{|l | c | c | c | c|}
\hline
& \textcolor{green}{Author} & \textcolor{blue}{Assessor 1} & \textcolor{magenta}{Assessor 2} & Total \\
\hline 
Safety function isolation (D.2.6-X-17)  & 1& & &  \\
\hline 
Safety properties formalisation (D.2.6-X-22)  &2 & & &  \\
\hline
Logical expression (D.2.6-X-28.2.2)  &2 & & &  \\
\hline
Timing constraints (D.2.6-X-28.2.3)  &3 & & &  \\
\hline
Safety properties validation (D.2.6-X-23.2)  &1 & & &  \\
\hline
Logical properties assertion (D.2.6-X-34)  &2 & & &  \\
\hline
Check  of assertions (D.2.6-X-34.1)  &1 & & &  \\
\hline
\end{tabular}
\begin{author_comment}
SysML formalism may be used to add constraints representing assertion
or properties. This can then be check by plug-in or external tools
During the modeling activities I did not try these requirements, I can
not say much about it.
\end{author_comment}
Does the language allow to  formalize (D.2.6-X-31):

\begin{tabular}{|l | c | c | c | c|}
\hline
& \textcolor{green}{Author} & \textcolor{blue}{Assessor 1} & \textcolor{magenta}{Assessor 2} & Total \\
\hline 
State machines  &3 & & &  \\
\hline
Time-outs  &3 & & &  \\
\hline
Truth tables  &1 & & &  \\
\hline
Arithmetic  &2 & & &  \\
\hline
Braking curves  &2 & & &  \\
\hline
Logical statements &3 & & &  \\
\hline
Message and fields &3 & & &  \\
\hline
\end{tabular}

\paragraph{Additional comments on semi-formal  model} Do you think your semi-formal  model is sufficient to cover a safe design of the on-board unit until code generation ?
All comments on links to  other models, validation and verification activities are welcomed.
\begin{author_comment}
SysML \cite{sysmlbook} is a good candidate for theses purposes.
\end{author_comment}
\subsection{Strictly formal model}

Concerning strictly formal model, how the WP2 requirements are covered ?

\begin{tabular}{|l | c | c | c | c|}
\hline
& \textcolor{green}{Author} & \textcolor{blue}{Assessor 1} & \textcolor{magenta}{Assessor 2} & Total \\
\hline 
Consistency to SFM (D.2.6-X-14.2) &3 & & &  \\
\hline
Coverage of SSRS (D.2.6-X-14.2)  & & 3& &  \\
\hline
Traceability to  SSRS (D.2.6-X-14.3)  & 3& & &  \\
\hline
Extensible to software design (D.2.6-X-16)  &3 & & &  \\
\hline
Safety function isolation (D.2.6-X-17)  &*  & & &  \\
\hline 
Safety properties formalization (D.2.6-X-22)  &3* & & &  \\
\hline
Logical expression (D.2.6-X-28.2.2)  &2 & & &  \\
\hline
Timing constraints (D.2.6-X-28.2.3)  &3 & & &  \\
\hline
Safety properties validation (D.2.6-X-23.3)  &1 & & &  \\
\hline
Logical properties assertion (D.2.6-X-34)  &2 & & &  \\
\hline
Proof of assertions (D.2.6-X-34.2)  &1 & & &  \\
\hline
\end{tabular}

\begin{author_comment}
\begin{itemize}
\item Safety function isolation :  I do not know.
\item Safety properties formalization : I did not try
\end{itemize}

\end{author_comment}
Does the language allow to  formalize (D.2.6-X-32):

\begin{tabular}{|l | c | c | c | c|}
\hline
& \textcolor{green}{Author} & \textcolor{blue}{Assessor 1} & \textcolor{magenta}{Assessor 2} & Total \\
\hline 
State machines  &3 & & &  \\
\hline
Time-outs  &3 & & &  \\
\hline
Truth tables  &1 & & &  \\
\hline
Arithmetic  & 2& & &  \\
\hline
Braking curves  &2 & & &  \\
\hline
Logical statements &3 & & &  \\
\hline
Message and fields &3 & & &  \\
\hline
\end{tabular}
\begin{author_comment}

The formal semantics of SysML is described in textual form in the UML and SysML standards
(see \url{http://www.omg.org/spec/UML/2.3/Superstructure/PDF/}, \url{http://www.omg.org/spec/SysML/1.2/PDF/})

The Chapters about State Machines in the UML
document gives a rather clear description (thought not really easy
to read ...) about the intended behaviors of state machines and the
semantic variation points that are NOT fixed by the standard, but may
be interpreted in different ways. More mathematical descriptions can
be found in  many research papers.
\end{author_comment}
\paragraph{Additional comments on semi-formal  model} Do you think your strictly formal  model can be directly defined from the SSRS ?
All comments on links to  other models, validation and verification activities are welcomed.

\begin{author_comment}
SysML is a good candidate for these purposes
\end{author_comment}
\section{Software design}
This section discusses the usage of the approach for software design.
It can be skipped depending the results of \ref{main_usage}.

\subsection{Functional design}

How the approach allows to  produce a functional software model of the on-board unit ?

\begin{tabular}{|l | c | c | c | c|}
\hline
& \textcolor{green}{Author} & \textcolor{blue}{Assessor 1} & \textcolor{magenta}{Assessor 2} & Total \\
\hline
Derivation from system semi-formal model  &3 & & &  \\
\hline 
Software architecture description  &3 & & &  \\
\hline
Software constraints  &3 & & &  \\
\hline
Traceability  &3 & & &  \\
\hline
Executable  &2 & & &  \\
\hline
\end{tabular}

\subsection{SSIL4 design}

How the approach allows to  produce in safety a software model ?

\begin{tabular}{|l | c | c | c | c|}
\hline
& \textcolor{green}{Author} & \textcolor{blue}{Assessor 1} & \textcolor{magenta}{Assessor 2} & Total \\
\hline
Derivation from system semi-formal or strictly formal model  &3 & & &  \\
\hline 
Software architecture description  &3 & & &  \\
\hline
Software constraints  &3 & & &  \\
\hline
Traceability  &3 & & &  \\
\hline
Executable  &2 & & &  \\
\hline
Conformance to EN50128 § 7.2  &3 & & &  \\
\hline
Conformance to EN50128 § 7.3  &3 & & &  \\
\hline
Conformance to EN50128 § 7.4  & 3& & &  \\
\hline
\end{tabular}

Which criteria for software architecture are covered by the methodology
(see EN50128 table A.3) :

\begin{tabular}{|l | c | c | c | c|}
\hline
& \textcolor{green}{Author} & \textcolor{blue}{Assessor 1} & \textcolor{magenta}{Assessor 2} & Total \\
\hline
Defensive programming  &* & & &  \\
\hline 
Fault detection \& diagnostic  &* & & &  \\
\hline
Error detecting code  &* & & &  \\
\hline
Failure assertion programming &3 & & &  \\
\hline
Diverse programming &3 & & &  \\
\hline
Memorising executed cases &* & & &  \\
\hline
Software error effect analysis &* & & &  \\
\hline
Fully defined interface &3 & & &  \\
\hline
Modelling  &3 & & &  \\
\hline
Structured methodology &3 & & &  \\
\hline
\end{tabular}
\begin{author_comment}
SysML is a modeling language, it can also be used to design
software. EA first purpose is not to provide a software development
infrastructure. This can be realized through plug-in or with external
tools.
EA itself does not provide mechanism for error detecting code ...
\end{author_comment}

\section{Software code generation}
This section discusses the usage of the approach for software code generation.
It can be skipped depending the results of \ref{main_usage}.

Which criteria for software design and implementation are covered by the methodology
(see EN50128 table A.4) :

\begin{tabular}{|l | c | c | c | c|}
\hline
& \textcolor{green}{Author} & \textcolor{blue}{Assessor 1} & \textcolor{magenta}{Assessor 2} & Total \\
\hline
Formal methods  &* & & &  \\
\hline 
Modeling  &3 & & &  \\
\hline
Modular approach (mandatory) &3 & & &  \\
\hline
Components &3 & & &  \\
\hline
Design and coding standards (mandatory) &3 & & &  \\
\hline
Strongly typed programming language &3 & & &  \\
\hline

\end{tabular}



\section{Main usage of the tool}
\label{main_usage}

This section discusses the main usage of the tool.

Which task are covered by the tool ?


\begin{tabular}{|l | c | c | c | c|}
\hline
& \textcolor{green}{Author} & \textcolor{blue}{Assessor 1} & \textcolor{magenta}{Assessor 2} & Total \\
\hline 
Modelling support &3 & & &  \\
\hline
Automatic translation  &3 & & & \\
\hline
Code Generation  &2 & & & \\
\hline
Model verification &1 & & & \\
\hline
Test generation &1 & & & \\
\hline
Simulation, execution, debugging &1 & & & \\
\hline
Formal proof &1 & & & \\
\hline
\end{tabular}

\paragraph{Modelling support}
Does the tool provide a  textual or a graphical editor ? Graphical editor.

\paragraph{Automatic translation and code generation}
Which translation or code generation is supported by the tool ?
\begin{itemize}
\item XMI
\item C++
\item Java ...
\end{itemize}
\paragraph{Model verification}
Which verification on models are provided by the tool?
None on the basic version of the tool.
\paragraph{Test generation}
Does the tool allow to generate tests ? For  which purpose ?
Not the basic version.
\paragraph{Simulation, execution, debugging}
Does the tool allow to simulate or to debbug step by step a model or a code ?
Not the basic version.
\paragraph{Formal proof}
Does the tool allow formal proof ?  How ?
Not the basic version.


\section{Use of the tool}


According WP2 requirements, give a note for characteristics of the use of the tool (from 0 to 3) :

\begin{tabular}{|l | c | c | c | c|}
\hline
& \textcolor{green}{Author} & \textcolor{blue}{Assessor 1} & \textcolor{magenta}{Assessor 2} & Total \\
\hline 
Open Source (D2.6-X-36) &0 & & &  \\
\hline 
Portability to operating systems (D2.6-X-37) &2 & & &  \\
\hline
Cooperation of tools (D2.6-X-38) &2 & & &  \\
\hline
Robustness (D2.6-X-41) &3 & & & \\
\hline
Modularity (D2.6-X-41.1) &3 & & & \\
\hline
Documentation management (D.2.6-X-41.2) & 2& & & \\
\hline
Distributed software development (D.2.6-X-41.3)  &3 & & & \\
\hline
Simultaneous multi-users (D.2.6-X-41.4)   &1 & & & \\
\hline
Issue tracking (D.2.6-X-41.5) &2 & & & \\
\hline
Differences between models (D.2.6-X-41.6) &1 & & & \\
\hline
Version management (D.2.6-X-41.7) &3 & & & \\
\hline
Concurrent version development (D.2.6-X-41.8) &1 & & & \\
\hline
Model-based version control (D.2.6-X-41.9) &0 & & & \\
\hline
Role traceability (D.2.6-X-41.10) & 3& & & \\
\hline
Safety version traceability (D.2.6-X-41.11) &0 & & & \\
\hline
Model traceability (D.2.6-01-035) &3 & & & \\
\hline
Tool chain integration &2* & & & \\
\hline
Scalability &3 & & & \\
\hline
\end{tabular}
\begin{author_comment}
For the tool chain integration it depends on the integration methods
choosen. If it is only by exchanging XMI file, there is no problem.
\end{author_comment}
\section{Certifiability}

This section discusses how the tool can be classified according EN50128 requirements (D.2.6-X-50).


\begin{tabular}{|l | c | c | c | c|}
\hline
& \textcolor{green}{Author} & \textcolor{blue}{Assessor 1} & \textcolor{magenta}{Assessor 2} & Total \\
\hline 
Tool manual (D.2.6-01-42.02) &3 & & &  \\
\hline
Proof of correctness (D.2.6-01-42.03)   &2 & & & \\
\hline
Existing industrial  usage  &3 & & & \\
\hline
Model verification &1 & & & \\
\hline
Test generation &0 & & & \\
\hline
Simulation, execution, debugging &1 & & & \\
\hline
Formal proof &0 & & & \\
\hline
\end{tabular}

\paragraph{Other elements for tool certification}

\section{Other comments}
Please to  give free comments on the approach.



