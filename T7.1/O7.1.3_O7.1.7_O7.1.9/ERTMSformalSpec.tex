\chapter{ERTMSFormalSpecs}

\begin{description}
\item[\textcolor{green}{Author}] Author of the approaches description  Stanislas Pinte (ERTMS Solutions)
\item[\textcolor{blue}{Assessor 1}] First assessor of the approaches Renaud De Landtsheer (Alstom Be)
\item[\textcolor{magenta}{Assessor 2}] Second assessor of the approaches Marielle Petit-Doche (Systerel)
\end{description}

In the sequel, main text is under the responsibilities of the author.

\begin{author_comment}
Author can add comments using this format at any place.
\end{author_comment}

\begin{assessor1}
First assessor can add comments using this format at any place.
\end{assessor1}

\begin{assessor2}
Second assessor can add comments using this format at any place.
\end{assessor2}

When a note is required, please follow this list :
\begin{description}
\item[0] not recommended, not adapted, rejected
\item[1] weakly recommended, adapted after major improvements, weakly rejected
\item[2] recommended, adapted (with light improvements if necessary)  weakly accepted
\item[3] highly recommended, well adapted,strongly accepted
\item[*] difficult to evaluate with a note (please add a comment under the table)
\end{description}

All the notes can be commented under each table.

\section{Presentation}

This section gives a quick presentation of the approach and the tool.

\begin{description}
\item[Name] ERTMSFormalSpecs
\item[Web site] https://www.ertmssolutions.com/ertms-formalspecs/
\item[Licence] EUPL (https://github.com/openETCS/ERTMSFormalSpecs)
\end{description}

\paragraph{Abstract} Short abstract on the approach and tool (10 lines max)

ERTMSFormalSpecs provides a domain-specific language, designed to express the ERTMS specification in a concise and verifiable formal representation. It is understandable by domain specialists while retaining the ability to be translated to executable representations by fully automated means.
\paragraph{Publications} Short list of publications on the approach (5 max)

http://www.ertmssolutions.com/files/ERTMSFormalSpecs_WCRR2011.pdf

http://www.ertmssolutions.com/files/UsingERTMSFormalSpecsToModelBrakingCurves.pdf

\section{Main usage of the approach}
\label{main_usage}
This section discusses the main usage of the approach.

According to the figure \ref{fig:main_process}, for which phases do you recommend the approach (give a note from 0 to  3) :

\begin{tabular}{|l | c | c | c | c|}
\hline
& \textcolor{green}{Author} & \textcolor{blue}{Assessor 1} & \textcolor{magenta}{Assessor 2} & Total \\
\hline 
System Analysis & \textcolor{green}{0} & & &  \\
\hline
Sub-system formal design & \textcolor{green}{3} & & & \\
\hline
Software design & \textcolor{green}{0} & & & \\
\hline
Software code generation & \textcolor{green}{0} & & & \\
\hline
\end{tabular}

\begin{author_comment}
The scope of ERTMSFormalSpecs is, as described above, "a domain-specific language, designed to express the ERTMS specification in a concise and verifiable formal representation."
Therefore, it is targeted to be used for sub-system formal design, not for analysis, software design or software code generation.
According to the figure \ref{fig:main_process}, for which type of activities do you recommend the approach (give a note from 0 to  3) :

\begin{tabular}{|l | c | c | c | c|}
\hline
& \textcolor{green}{Author} & \textcolor{blue}{Assessor 1} & \textcolor{magenta}{Assessor 2} & Total \\
\hline 
Documentation & \textcolor{green}{3} & & &  \\
\hline
Modeling & \textcolor{green}{3} & & &  \\
\hline
Design & \textcolor{green}{3} & & & \\
\hline
Code generation & \textcolor{green}{0} & & & \\
\hline
Verification & \textcolor{green}{3} & & & \\
\hline
Validation & \textcolor{green}{3} & & & \\
\hline
Safety analyses & \textcolor{green}{0} & & & \\
\hline
\end{tabular}

\paragraph{Known usages} Have you some examples of usage of this approach to  compare with the OpenETCS objectives ?

The ERTMSFormalSpecs approach has been designed specifically for the purpose of modelling an OBU application software. The ERTMSFormalSpecs approach is 100\% aligned with the OpenETCS objective 1, which is to have a 100\% semi-formal model of the SSRS. 
\section{Language}
This section discusses the main element of the language.

Which are the main characteristics of the language :

\begin{tabular}{|l | c | c | c | c|}
\hline
& \textcolor{green}{Author} & \textcolor{blue}{Assessor 1} & \textcolor{magenta}{Assessor 2} & Total \\
\hline 
Informal language & \textcolor{green}{0} & & &  \\
\hline 
Semi-formal language & \textcolor{green}{3} & & &  \\
\hline
Formal language & \textcolor{green}{3} & & &  \\
\hline
Structured language & \textcolor{green}{3} & & & \\
\hline
Modular language & \textcolor{green}{3} & & & \\
\hline
Textual language & \textcolor{green}{3} & & & \\
\hline
Mathematical symbols or code & \textcolor{green}{3} & & & \\
\hline
Graphical language & \textcolor{green}{3} & & & \\
\hline
\end{tabular}
According WP2 requirements, give a note for the capabilities of the language (from 0 to 3) :

\begin{tabular}{|l | c | c | c | c|}
\hline
& \textcolor{green}{Author} & \textcolor{blue}{Assessor 1} & \textcolor{magenta}{Assessor 2} & Total \\
\hline
Declarative formalization of properties (D.2.6-X-28) & \textcolor{green}{0} & & & \\
\hline
Simple formalization of properties (D.2.6-X-28.1) & \textcolor{green}{2} & & & \\
\hline
Scalability : capability to design large model & \textcolor{green}{3} & & & \\
\hline
Easily translatable to other languages (D.2.6-X-30) & \textcolor{green}{3} & & & \\
\hline
Executable directly (D.2.6-X-33) & \textcolor{green}{3} & & & \\
\hline
Executable after translation to a code (D.2.6-X-33) & \textcolor{green}{3} & & & \\
(precise if the translation is automatic) & \textcolor{green}{2} & & & \\
\hline
Simulation, animation (D.2.6-X-33) & \textcolor{green}{3} & & & \\
\hline
Easily understandable (D.2.6-X-27) & \textcolor{green}{3} & & & \\
\hline
Expertise level needed (0 High level, 3 few level) & \textcolor{green}{2} & & & \\
\hline
Standardization (D.2.6-X-29) & \textcolor{green}{3} & & & \\
\hline
Documented (D.2.6-X-29) & \textcolor{green}{3} & & & \\
\hline
Extensible language (D.2.6-01-28) & \textcolor{green}{3} & & & \\
\hline
\end{tabular}


\paragraph{Documentation} Describe how the language is documented, the existing guidelines, coding rules, standardization...

ERTMSFormalSpecs provides the following documentation set:

\begin{itemize}
	\item EFSW_Release_Notes.pdf (https://github.com/openETCS/ERTMSFormalSpecs/blob/master/ErtmsFormalSpecs/doc/EFSW_Release_Notes.pdf)
	\item EFSW_Technical_Design.pdf (https://github.com/openETCS/ERTMSFormalSpecs/blob/master/ErtmsFormalSpecs/doc/EFSW_Technical_Design.pdf)
	\item EFSW_User_Guide.pdf (https://github.com/openETCS/ERTMSFormalSpecs/blob/master/ErtmsFormalSpecs/doc/EFSW_User_Guide.pdf)
	\item ERTMSFormalSpecs-Tutorial (https://github.com/openETCS/ERTMSFormalSpecs/wiki/ERTMSFormalSpecs-Tutorial)
	\item ERTMSFormalSpecs-FAQ (https://github.com/openETCS/ERTMSFormalSpecs/wiki/ERTMSFormalSpecs-FAQ)
\end{itemize}


\paragraph{Language usage} Describe the possible restriction on the language



\section{System Analysis}
This section discusses the usage of the approach for system analysis.
It can be skipped depending the results of \ref{main_usage}.

Acoording WP2 requirements, how the approach can be involved for the sub-system requirement specification ?

\begin{tabular}{|l | c | c | c | c|}
\hline
& \textcolor{green}{Author} & \textcolor{blue}{Assessor 1} & \textcolor{magenta}{Assessor 2} & Total \\
\hline
Independent System functions definition (D.2.6-X-10.2.1)  & \textcolor{green}{3} & & &  \\
\hline 
System architecture design (D.2.6-X-10.2) & \textcolor{green}{3} & & &  \\
\hline
System data flow identification (D.2.6-X-10.2.3)  & \textcolor{green}{3} & & &  \\
\hline
Sub-system focus (D.2.6-X-10.2.4)  & \textcolor{green}{3} & & &  \\
\hline
System interfaces definition (D.2.6-X-10.2.5)  & \textcolor{green}{3} & & &  \\
\hline
System requirement allocation (D.2.6-X-10.3)  & \textcolor{green}{3} & & &  \\
\hline
Traceability with SRS (D.2.6-X-10.5)  & \textcolor{green}{3} & & &  \\
\hline
Traceability with Safety activities (D.2.6-X-11)  & \textcolor{green}{0} & & &  \\
\hline
\end{tabular}

\begin{author_comment}
Although ERTMSFormalSpecs is not made for system analysis, it can be used for the following aspects on system level: Modelling of (separate) system functions, data flows, state machines and interfaces. It also provides complete SRS traceability support.  
\end{author_comment}


\section{Sub-System formal design}
This section discusses the usage of the approach for sub-system formal design.
It can be skipped depending the results of \ref{main_usage}.

Two kinds of model can be planned during this phase: semi-formal models to  cover the SSRS (D.2.6-X-12.1) and strictly formal  models to  focuss on some functional and safety aspects (D.2.6-X-14).  Obviously some strictly  formal means can be used to define the semi-formal  model.

\subsection{Semi-formal model}

\begin{author_comment}
ERTMSFormalSpecs models are formal in the sense that the ERTMSFormalSpecs language is fully defined with a grammar and complete semantics. However, they are semi-formal in the sense that there is no mathematical proof theory at the basis of the language definition.  
\end{author_comment}


Concerning semi-formal model, how the WP2 requirements are covered ?

\begin{tabular}{|l | c | c | c | c|}
\hline
& \textcolor{green}{Author} & \textcolor{blue}{Assessor 1} & \textcolor{magenta}{Assessor 2} & Total \\
\hline 
Consistency to SSRS (D.2.6-X-12.2) & \textcolor{green}{3} & & &  \\
\hline
Coverage of SSRS (D.2.6-X-12.2.1)  & \textcolor{green}{3} & & &  \\
\hline
Coverage of SSHA (D.2.6-X-12.2.2)  & \textcolor{green}{3} & & &  \\
\hline
Management of requirement justification (D.2.6-X-12.2.3)  & \textcolor{green}{3} & & &  \\
\hline
Traceability to  SSRS (D.2.6-X-12.2.5)  & \textcolor{green}{3} & & &  \\
\hline
Traceability of exported requirements (D.2.6-X-12.2.6)  & \textcolor{green}{3} & & &  \\
\hline
Simulation or animation (D.2.6-X-13 partial)  & \textcolor{green}{3} & & &  \\
\hline
Execution (D.2.6-X-13 partial)  & \textcolor{green}{3} & & &  \\
\hline
Extensible to strictly formal model (D.2.6-X-14.3) & \textcolor{green}{3} & & &  \\
\hline
Easy to  refine towards strictly formal model (D.2.6-X-14.4) & \textcolor{green}{3} & & &  \\
\hline
Extensible and modular design (D.2.6-X-15)  & \textcolor{green}{3} & & &  \\
\hline
Extensible to software architecture and design (D.2.6-X-30)   & \textcolor{green}{3} & & &  \\
\hline
\end{tabular}

Concerning safety properties management, how the WP2 requirements are covered ?

\begin{tabular}{|l | c | c | c | c|}
\hline
& \textcolor{green}{Author} & \textcolor{blue}{Assessor 1} & \textcolor{magenta}{Assessor 2} & Total \\
\hline 
Safety function isolation (D.2.6-X-17)  & \textcolor{green}{1} & & &  \\
\hline 
Safety properties formalisation (D.2.6-X-22)  & \textcolor{green}{2} & & &  \\
\hline
Logical expression (D.2.6-X-28.2.2)  & \textcolor{green}{2} & & &  \\
\hline
Timing constraints (D.2.6-X-28.2.3)  & \textcolor{green}{2} & & &  \\
\hline
Safety properties validation (D.2.6-X-23.2)  & \textcolor{green}{2} & & &  \\
\hline
Logical properties assertion (D.2.6-X-34)  & \textcolor{green}{2} & & &  \\
\hline
Check  of assertions (D.2.6-X-34.1)  & \textcolor{green}{2} & & &  \\
\hline
\end{tabular}

\begin{author_comment}
ERTMSFormalSpecs is a modelling language for functions. Therefore, only the functional aspects of properties are addressed.  
\end{author_comment}
Does the language allow to  formalize (D.2.6-X-31):

\begin{tabular}{|l | c | c | c | c|}
\hline
& \textcolor{green}{Author} & \textcolor{blue}{Assessor 1} & \textcolor{magenta}{Assessor 2} & Total \\
\hline 
State machines  & \textcolor{green}{3} & & &  \\
\hline
Time-outs  & \textcolor{green}{3} & & &  \\
\hline
Truth tables  & \textcolor{green}{3} & & &  \\
\hline
Arithmetic  & \textcolor{green}{3} & & &  \\
\hline
Braking curves  & \textcolor{green}{3} & & &  \\
\hline
Logical statements & \textcolor{green}{3} & & &  \\
\hline
Message and fields & \textcolor{green}{3} & & &  \\
\hline
\end{tabular}

\paragraph{Additional comments on semi-formal  model} Do you think your semi-formal  model is sufficient to cover a safe design of the on-board unit until code generation ?
All comments on links to  other models, validation and verification activities are welcomed.

\begin{author_comment}
ERTMSFormalSpecs has been designed to model the Subset-026 and test the S026 model, without taking Safety aspects into account initially.    
\end{author_comment}
\subsection{Strictly formal model}

Concerning strictly formal model, how the WP2 requirements are covered ?

\begin{author_comment}
Even though ERTMSFormalSpecs models are formal, ERTMSFormalSpecs doesn't aim to be used a strictly formal model, for proving purposes, in the context of the OpenETCS project. Therefore that section is skipped from evaluation.  
\end{author_comment}

\begin{tabular}{|l | c | c | c | c|}
\hline
& \textcolor{green}{Author} & \textcolor{blue}{Assessor 1} & \textcolor{magenta}{Assessor 2} & Total \\
\hline 
Consistency to SFM (D.2.6-X-14.2) & & & &  \\
\hline
Coverage of SSRS (D.2.6-X-14.2)  & & & &  \\
\hline
Traceability to  SSRS (D.2.6-X-14.3)  & & & &  \\
\hline
Extensible to software design (D.2.6-X-16)  & & & &  \\
\hline
Safety function isolation (D.2.6-X-17)  & & & &  \\
\hline 
Safety properties formalisation (D.2.6-X-22)  & & & &  \\
\hline
Logical expression (D.2.6-X-28.2.2)  & & & &  \\
\hline
Timing constraints (D.2.6-X-28.2.3)  & & & &  \\
\hline
Safety properties validation (D.2.6-X-23.3)  & & & &  \\
\hline
Logical properties assertion (D.2.6-X-34)  & & & &  \\
\hline
Proof of assertions (D.2.6-X-34.2)  & & & &  \\
\hline
\end{tabular}

Does the language allow to  formalize (D.2.6-X-32):

\begin{tabular}{|l | c | c | c | c|}
\hline
& \textcolor{green}{Author} & \textcolor{blue}{Assessor 1} & \textcolor{magenta}{Assessor 2} & Total \\
\hline 
State machines  & & & &  \\
\hline
Time-outs  & & & &  \\
\hline
Truth tables  & & & &  \\
\hline
Arithmetic  & & & &  \\
\hline
Braking curves  & & & &  \\
\hline
Logical statements & & & &  \\
\hline
Message and fields & & & &  \\
\hline
\end{tabular}

\paragraph{Additional comments on semi-formal  model} Do you think your strictly formal  model can be directly defined from the SSRS ?
All comments on links to  other models, validation and verification activities are welcomed.


\section{Software design}
This section discusses the usage of the approach for software design.
It can be skipped depending the results of \ref{main_usage}.

\begin{author_comment}
ERTMSFormalSpecs scope is limited to modelling in the large (modelling, test and documentation). Therefore, software design section is skipped.  
\end{author_comment}

\subsection{Functional design}

How the approach allows to  produce a functional software model of the on-board unit ?

\begin{tabular}{|l | c | c | c | c|}
\hline
& \textcolor{green}{Author} & \textcolor{blue}{Assessor 1} & \textcolor{magenta}{Assessor 2} & Total \\
\hline
Derivation from system semi-formal model  & & & &  \\
\hline 
Software architecture description  & & & &  \\
\hline
Software constraints  & & & &  \\
\hline
Traceability  & & & &  \\
\hline
Executable  & & & &  \\
\hline
\end{tabular}

\subsection{SSIL4 design}

How the approach allows to  produce in safety a software model ?

\begin{tabular}{|l | c | c | c | c|}
\hline
& \textcolor{green}{Author} & \textcolor{blue}{Assessor 1} & \textcolor{magenta}{Assessor 2} & Total \\
\hline
Derivation from system semi-formal or strictly formal model  & & & &  \\
\hline 
Software architecture description  & & & &  \\
\hline
Software constraints  & & & &  \\
\hline
Traceability  & & & &  \\
\hline
Executable  & & & &  \\
\hline
Conformance to EN50128 § 7.2  & & & &  \\
\hline
Conformance to EN50128 § 7.3  & & & &  \\
\hline
Conformance to EN50128 § 7.4  & & & &  \\
\hline
\end{tabular}

Which criteria for software architecture are covered by the methodology
(see EN50128 table A.3) :

\begin{tabular}{|l | c | c | c | c|}
\hline
& \textcolor{green}{Author} & \textcolor{blue}{Assessor 1} & \textcolor{magenta}{Assessor 2} & Total \\
\hline
Defensive programming  & & & &  \\
\hline 
Fault detection \& diagnostic  & & & &  \\
\hline
Error detecting code  & & & &  \\
\hline
Failure assertion programming & & & &  \\
\hline
Diverse programming & & & &  \\
\hline
Memorising executed cases & & & &  \\
\hline
Software error effect analysis & & & &  \\
\hline
Fully defined interface & & & &  \\
\hline
Modelling  & & & &  \\
\hline
Structured methodology & & & &  \\
\hline
\end{tabular}

\section{Software code generation}
This section discusses the usage of the approach for software code generation.
It can be skipped depending the results of \ref{main_usage}.

\begin{author_comment}
ERTMSFormalSpecs scope is limited to modelling in the large (modelling, test and documentation). Therefore, software code generation section is skipped.  
\end{author_comment}

Which criteria for software design and implementation are covered by the methodology
(see EN50128 table A.4) :

\begin{tabular}{|l | c | c | c | c|}
\hline
& \textcolor{green}{Author} & \textcolor{blue}{Assessor 1} & \textcolor{magenta}{Assessor 2} & Total \\
\hline
Formal methods  & & & &  \\
\hline 
Modeling  & & & &  \\
\hline
Modular approach (mandatory) & & & &  \\
\hline
Components & & & &  \\
\hline
Design and coding standards (mandatory) & & & &  \\
\hline
Strongly typed programming language & & & &  \\
\hline

\end{tabular}



\section{Main usage of the tool}
\label{main_usage}

This section discusses the main usage of the tool.

Which task are covered by the tool ?


\begin{tabular}{|l | c | c | c | c|}
\hline
& \textcolor{green}{Author} & \textcolor{blue}{Assessor 1} & \textcolor{magenta}{Assessor 2} & Total \\
\hline 
Modelling support & \textcolor{green}{3} & & &  \\
\hline
Automatic translation  & \textcolor{green}{1} & & & \\
\hline
Code Generation  & \textcolor{green}{1} & & & \\
\hline
Model verification & \textcolor{green}{3} & & & \\
\hline
Test generation & \textcolor{green}{1} & & & \\
\hline
Simulation, execution, debugging & \textcolor{green}{3} & & & \\
\hline
Formal proof & \textcolor{green}{0} & & & \\
\hline
\end{tabular}

\paragraph{Modelling support}
Does the tool provide a  textual or a graphical editor ?

Both.

\paragraph{Automatic translation and code generation}
Which translation or code generation is supported by the tool ?
As of today, the ERTMSFormalSpecs model is available both as an XML file and as an EMF model. The EMF model can be used to develop model-to-model translator (for instance ERTMSFormalSpecs->SCADE) or code generators.

\paragraph{Model verification}
Which verification on models are provided by the tool?

The ERTMSFormalSpecs model can be tested, by writing ERTMSFormalSpecs test cases in ERTMSFormalSpecs language (based on steps, actions and expectations), executing and debugging these test cases, and generating a test report. 

ERTMSFormalSpecs test cases can also be executed in an automated fashion for nightly build non-regression testing purposes.

\paragraph{Test generation}
Does the tool allow to generate tests ? For  which purpose ?

No, ERTMSFormalSpecs doesn't allow to generate tests. Integration between ERTMSFormalSpecs and RT-Tester is under study, to allow for automatic ERTMSFormalSpecs model test cases generation.

\paragraph{Simulation, execution, debugging}
Does the tool allow to simulate or to debbug step by step a model or a code ?

Yes, all of this is described in the ERTMSFormalSpecs User Guide.

\paragraph{Formal proof}
Does the tool allow formal proof ?  How ?
No, ERTMSFormalSpecs doesn't allow formal proof. 


\section{Use of the tool}


According WP2 requirements, give a note for characteristics of the use of the tool (from 0 to 3) :

\begin{tabular}{|l | c | c | c | c|}
\hline
& \textcolor{green}{Author} & \textcolor{blue}{Assessor 1} & \textcolor{magenta}{Assessor 2} & Total \\
\hline 
Open Source (D2.6-X-36) & \textcolor{green}{3} & & &  \\
\hline 
Portability to operating systems (D2.6-X-37) & \textcolor{green}{2} & & &  \\
\hline
Cooperation of tools (D2.6-X-38) & \textcolor{green}{3} & & &  \\
\hline
Robustness (D2.6-X-41) & \textcolor{green}{3} & & & \\
\hline
Modularity (D2.6-X-41.1) & \textcolor{green}{3} & & & \\
\hline
Documentation management (D.2.6-X-41.2) & \textcolor{green}{2} & & & \\
\hline
Distributed software development (D.2.6-X-41.3)  & \textcolor{green}{2} & & & \\
\hline
Simultaneous multi-users (D.2.6-X-41.4)   & \textcolor{green}{1} & & & \\
\hline
Issue tracking (D.2.6-X-41.5) & \textcolor{green}{3} & & & \\
\hline
Differences between models (D.2.6-X-41.6) & \textcolor{green}{1} & & & \\
\hline
Version management (D.2.6-X-41.7) & \textcolor{green}{2} & & & \\
\hline
Concurrent version development (D.2.6-X-41.8) & \textcolor{green}{2} & & & \\
\hline
Model-based version control (D.2.6-X-41.9) & \textcolor{green}{2} & & & \\
\hline
Role traceability (D.2.6-X-41.10) & \textcolor{green}{1} & & & \\
\hline
Safety version traceability (D.2.6-X-41.11) & \textcolor{green}{0} & & & \\
\hline
Model traceability (D.2.6-01-035) & \textcolor{green}{3} & & & \\
\hline
Tool chain integration & \textcolor{green}{2} & & & \\
\hline
Scalability & \textcolor{green}{3} & & & \\
\hline
\end{tabular}

\section{Certifiability}

This section discusses how the tool can be classified according EN50128 requirements (D.2.6-X-50).

\begin{author_comment}
ERTMSFormalSpecs has EN50128 certifiability compliance outside of its scope, for the version available as of today. Certifiability compliance may be in the scope of future versions. However, relevant sections of this section are filled.
\end{author_comment}


\begin{tabular}{|l | c | c | c | c|}
\hline
& \textcolor{green}{Author} & \textcolor{blue}{Assessor 1} & \textcolor{magenta}{Assessor 2} & Total \\
\hline 
Tool manual (D.2.6-01-42.02) & \textcolor{green}{3} & & &  \\
\hline
Proof of correctness (D.2.6-01-42.03)   & \textcolor{green}{2} & & & \\
\hline
Existing industrial  usage  &  \textcolor{green}{0} & & & \\
\hline
Model verification & \textcolor{green}{3} & & & \\
\hline
Test generation & \textcolor{green}{0} & & & \\
\hline
Simulation, execution, debugging & \textcolor{green}{3} & & & \\
\hline
Formal proof & \textcolor{green}{0} & & & \\
\hline
\end{tabular}

\paragraph{Other elements for tool certification}

\section{Other comments}
Please to  give free comments on the approach.



