\chapter{Classical B and Atelier B}
\label{chap:classicalB}


\begin{description}
\item[\textcolor{green}{Author}] Author of the approaches description  Marielle Petit-Doche (Systerel)
\item[\textcolor{blue}{Assessor 1}] First assessor of the approaches Peyman Farhangi (DB), Peter Mahlmann (DB)
\item[\textcolor{magenta}{Assessor 2}] Second assessor of the approaches Jan Welte (TU-BS)
\end{description}

In the sequel, main text is under the responsibilities of the author.

\begin{author_comment}
Author can add comments using this format at any place.
\end{author_comment}

\begin{assessor1}
First assessor can add comments using this format at any place.
\end{assessor1}

\begin{assessor2}
Second assessor can add comments using this format at any place.
\end{assessor2}

When a note is required, please follow this list :
\begin{description}
\item[0] not recommended, not adapted, rejected
\item[1] weakly recommended, adapted after major improvements, weakly rejected
\item[2] recommended, adapted (with light improvements if necessary)  weakly accepted
\item[3] highly recommended, well adapted,strongly accepted
\item[*] difficult to evaluate with a note (please add a comment under the table)
\end{description}

All the notes can be commented under each table.

\section{Presentation}

This section gives a quick presentation of the approach and the tool.

\begin{description}
\item[Name] Classical B and Atelier B
\item[Web site] \url{http://www.atelierb.eu/en/}
\item[Licence] Free but close licence (with progressive Open Source implementation of Atelier B tools)
\end{description}

\paragraph{Abstract} 

The B-Method is a formal method developed by J-R. Abrial and used in industry, especially in railway industry, to develop complex systems. It covers software development from formal specification to code level. Proof mechanisms guaranty the consistency of specifications properties and the complete consistency of code regarding its formal specification. It is efficient to model  functional  elements of a critical software with respect to  EN50128 constraints.

AtelierB is the industrial  tool the most used to develop critical software.

\paragraph{Publications} 
[Abrial1996] The B Book, Assigning Programs to Meanings


\section{Main usage of the approach}
\label{main_usage}
This section discusses the main usage of the approach.

According to the figure \ref{fig:main_process}, for which phases do you recommend the approach (give a note from 0 to  3) :

\begin{tabular}{|l | c | c | c | c|}
\hline
& \textcolor{green}{Author} & \textcolor{blue}{Assessor 1} & \textcolor{magenta}{Assessor 2} & Total \\
\hline 
System Analysis & 1    & 1    & 1    & 3    \\
\hline
Sub-system formal design & 2    & 2    & 1    & \textcolor{blue}{5} \\
\hline
Software design & 3    & 3    & 3    & \textcolor{red}{\textbf{9}} \\
\hline
Software code generation & 3    & 3    & 3    & \textcolor{red}{\textbf{9}} \\
\hline
\end{tabular}

\begin{author_comment}
Classical B can be used for system design, however it is most adapted to software design. Event-B  based on the same language is better for system analyses \ref{chap:eventB}.
\end{author_comment}


\begin{assessor2}
As classical B is using mathematical expressed conditions and operations to describe the system behaviour it is not suited to understand and present the overall system architecture. It is also difficult to communicate with domain experts based on a classical B model.
\end{assessor2}

According to the figure \ref{fig:main_process}, for which type of activities do you recommend the approach (give a note from 0 to  3) :

\begin{tabular}{|l | c | c | c | c|}
\hline
& \textcolor{green}{Author} & \textcolor{blue}{Assessor 1} & \textcolor{magenta}{Assessor 2} & Total \\
\hline 
Documentation & \textcolor{green}{0} & \textcolor{green}{0} & \textcolor{green}{0} & \textcolor{green}{0} \\
\hline
Modeling & 3    & 3    & 3    & \textcolor{red}{\textbf{9}} \\
\hline
Design & 3    & 3    & 2    & \textcolor{magenta}{8} \\
\hline
Code generation & 3    & 3    & 3    & \textcolor{red}{\textbf{9}} \\
\hline
Verification & 3    & 3    & 3    & \textcolor{red}{\textbf{9}} \\
\hline
Validation & 2    & 1    & 1    & 4    \\
\hline
Safety analyses & 1    & 1    & 1    & 3    \\
\hline
\end{tabular}


\begin{author_comment}
Classical B  can be used to  validate safety properties.
\end{author_comment}

\begin{assessor2}
As classical b is mainly a strictly formal modelling approach, allowing to use formal proofs for specific constrains. Respectively the strength of this models is to specify the software behaviour and demonstrated that already defined constraints (stated based on earlier system analysis) are holding in this specification. Thereby the abstraction levels are mainly on the lower software level.
\end{assessor2}

\paragraph{Known usages} Have you some examples of usage of this approach to  compare with the OpenETCS objectives ?

\begin{author_comment}

Classical B has been used successfully  in railway  industry (mainly by Alstom, Siemens and AREVA) to  develop critical software in urban (CBTC, PMI,...) and mainline domains (KVB, Eurobalise,...). Hundred of different systems are running in the world embedding software developed in B (see \url{http://www.cs.vu.nl/~wanf/pubs/handbookFFM.pdf}, \url{http://link.springer.com/chapter/10.1007/3-540-48119-2_22} and \url{http://web.tiscali.it/chiccoterri/Metod B.htm}).

It is also used in nuclear, aeronautic and defence area.

\end{author_comment}


\section{Language}
This section discusses the main element of the language.

Which are the main characteristics of the language :

\begin{tabular}{|l | c | c | c | c|}
\hline
& \textcolor{green}{Author} & \textcolor{blue}{Assessor 1} & \textcolor{magenta}{Assessor 2} & Total \\
\hline 
Informal language & \textcolor{green}{0} & \textcolor{green}{0} & \textcolor{green}{0} & \textcolor{green}{0} \\
\hline 
Semi-formal language & \textcolor{green}{0} & \textcolor{green}{0} & \textcolor{green}{0} & \textcolor{green}{0} \\
\hline
Formal language & 3    & 3    & 3    & \textcolor{red}{\textbf{9}} \\
\hline
Structured language & 3    & 3    & 3    & \textcolor{red}{\textbf{9}} \\
\hline
Modular language & 3    & 3    & 3    & \textcolor{red}{\textbf{9}} \\
\hline
Textual language & 1    & 1    & 1    & 3    \\
\hline
Mathematical symbols or code & 3    & 3    & 3    & \textcolor{red}{\textbf{9}} \\
\hline
Graphical language & 1    & 1    & \textcolor{green}{0} & 2    \\
\hline
\end{tabular}

According WP2 requirements, give a note for the capabilities of the language (from 0 to 3) :

\begin{tabular}{|l | c | c | c | c|}
\hline
& \textcolor{green}{Author} & \textcolor{blue}{Assessor 1} & \textcolor{magenta}{Assessor 2} & Total \\
\hline
Declarative formalization of properties (D2.6-02-066) & 3    & 3    & 3    & \textcolor{red}{\textbf{9}} \\
\hline
Simple formalization of properties (D2.6-02-066.01) & 3    & 3    & 3    & \textcolor{red}{\textbf{9}}  \\
\hline
Scalability : capability to design large model & 3    & 3    & 3    & \textcolor{red}{\textbf{9}} \\
\hline
Easily translatable to other languages (D2.6-02-068) & 3    & 3    & 3    & \textcolor{red}{\textbf{9}} \\
\hline
Executable directly (D2.6-02-071) & \textcolor{green}{0} & \textcolor{green}{0} & \textcolor{green}{0} & \textcolor{green}{0} \\
\hline
Executable after translation to a code (D2.6-02-071) & 3    & 3    & 3    & \textcolor{red}{\textbf{9}} \\
(precise if the translation is automatic) & & & & \\
\hline
Simulation, animation (D2.6-02-071) &  3 & 2    & 3    & \textcolor{magenta}{8} \\
\hline
Easily understandable (D2.6-02-065) & 2    & 2    & 2    & \textcolor{blue}{6} \\
\hline
Expertise level needed (0 High level, 3 few level) &  1 & 1    & 1    & 3    \\
\hline
Standardization (D2.6-02-067) & 3    & 3    & 3    & \textcolor{red}{\textbf{9}} \\
\hline
Documented (D2.6-02-067) & 3    & 3    & 3    & \textcolor{red}{\textbf{9}} \\
\hline
Extensible language (D.2.6-01-28) & 1    & * & 1    & 2   * \\
\hline
\end{tabular}


\paragraph{Documentation} Describe how the language is documented, the existing guidelines, coding rules, standardization...

\begin{author_comment}

Language is documented with language manual reference (\url{http://www.atelierb.eu/ressources/manrefb1.8.6.uk.pdf}), tool with the user manual (\url{http://www.tools.clearsy.com/resources/User_uk.pdf}). Industrial have developed their own coding rules and guidelines.
\end{author_comment}

\begin{assessor2}
Since classical B has been used in many industrial and research projects (primarily in France) various introductions, coding rules and guidelines exist. The B book and the AtelierB B manuals present a quasi standard, but no official standard exist.
\end{assessor2}

\paragraph{Language usage} Describe the possible restriction on the language
\begin{author_comment}
Some restrictions in the language manual reference. Industrial guides can add restrictions.
\end{author_comment}

\section{System Analysis}
This section discusses the usage of the approach for system analysis.
It can be skipped depending the results of \ref{main_usage}.

Acoording WP2 requirements, how the approach can be involved for the sub-system requirement specification ?

\begin{tabular}{|l | c | c | c | c|}
\hline
& \textcolor{green}{Author} & \textcolor{blue}{Assessor 1} & \textcolor{magenta}{Assessor 2} & Total \\
\hline
Independent System functions definition (D2.6-02-045.02.1)  & 3    & 3    & 3    & \textcolor{red}{\textbf{9}} \\
\hline 
System architecture design (D2.6-02-045.02) & 3    & 3    & 1    & \textcolor{magenta}{7} \\
\hline
System data flow identification (D2.6-02-045.02.3)  & 3    & 3    & 2    & \textcolor{magenta}{8} \\
\hline
Sub-system focus (D2.6-02-045.02.4)  & 3    & 3    & 3    & \textcolor{red}{\textbf{9}} \\
\hline
System interfaces definition (D2.6-02-045.02.5)  & 3    & 3    & 3    & \textcolor{red}{\textbf{9}} \\
\hline
System requirement allocation (D2.6-02-045.03)  & 2    & 1    & 2    & \textcolor{blue}{5} \\
\hline
Traceability with SRS (D2.6-02-045.05)  & 2    & 1    & 2    & \textcolor{blue}{5} \\
\hline
Traceability with Safety activities (D2.6-02-046)  & 2    & 2    & 1    & \textcolor{blue}{5} \\
\hline
\end{tabular}

\begin{author_comment}
Classical B can be used to  support partly  system analysis. However Event B, based on the same language, is more suitable \ref{chap:eventB}.
\end{author_comment}

\begin{assessor2}
Classical B is primarily designed to do software specifications and verify their properties.
\end{assessor2}

\section{Sub-System formal design}
This section discusses the usage of the approach for sub-system formal design.
It can be skipped depending the results of \ref{main_usage}.

Two kinds of model can be planned during this phase: semi-formal models to  cover the SSRS (D2.6-02-047.01) and strictly formal  models to  focuss on some functional and safety aspects (D2.6-02-049).  Obviously some strictly  formal means can be used to define the semi-formal  model.

\subsection{Semi-formal model}


\begin{author_comment}
As a formal language, classical B  can cover some artifacts of semi-formal models.
\end{author_comment}

\begin{assessor2}
Since classical B is a more low level of abstraction formal language, it is difficult to estimate, how it can be used for the more high level work, which first will be done by the semi-formal model. Classical B will definitely be able to work with the SSRS document and demonstrate the coverage, but I is not well suited to build a high level system model.
\end{assessor2}

Concerning semi-formal model, how the WP2 requirements are covered ?

\begin{tabular}{|l | c | c | c | c|}
\hline
& \textcolor{green}{Author} & \textcolor{blue}{Assessor 1} & \textcolor{magenta}{Assessor 2} & Total \\
\hline 
Consistency to SSRS (D2.6-02-047.02) & 2    & 2    & 1    & \textcolor{blue}{5} \\
\hline
Coverage of SSRS (D2.6-02-047.02.01)  & 3    & 3    & 2    & \textcolor{magenta}{8} \\
\hline
Coverage of SSHA (D2.6-02-047.02.02)  & 2    & 2    & 2    & \textcolor{blue}{6} \\
\hline
Management of requirement justification (D2.6-02-047.02.03)  & 1    & 1    & 1    & 3    \\
\hline
Traceability to  SSRS (D2.6-02-047.02.05)  & 2    & 2    & 2    & \textcolor{blue}{6} \\
\hline
Traceability of exported requirements (D2.6-02-047.02.06)  & 1    & * & 1    & 2   * \\
\hline
Simulation or animation (D2.6-02-048 partial)  & 2    & 1    & 2    &  5 \\
\hline
Execution (D2.6-02-048 partial)  & 1    & \textcolor{green}{0} & 2    & 3    \\
\hline
Extensible to strictly formal model (D2.6-02-049.3) & 3    & 3    & 3    & \textcolor{red}{\textbf{9}}  \\
\hline
Easy to  refine towards strictly formal model (D2.6-02-049.4) & 3    & 3    & 3    & \textcolor{red}{\textbf{9}}  \\
\hline
Extensible and modular design (D2.6-02-050)  & 3    & 3    & 3    & \textcolor{red}{\textbf{9}} \\
\hline
Extensible to software architecture and design (D2.6-02-068)   & 3    & 3    & 3    & \textcolor{red}{\textbf{9}} \\
\hline
\end{tabular}


\begin{author_comment}
Execution is possible after translation.
\end{author_comment}
\begin{assessor1}
* Not sure of the meaning of exported requirements.
\end{assessor1}

Concerning safety properties management, how the WP2 requirements are covered ?

\begin{tabular}{|l | c | c | c | c|}
\hline
& \textcolor{green}{Author} & \textcolor{blue}{Assessor 1} & \textcolor{magenta}{Assessor 2} & Total \\
\hline 
Safety function isolation (D2.6-02-052)  & 3    & 3    & 3    & \textcolor{red}{\textbf{9}} \\
\hline 
Safety properties formalisation (D2.6-02-057)  & 2    & 2    & 2    & \textcolor{blue}{6} \\
\hline
Logical expression (D2.6-02-066.02.01)  & 3    & 3    & 3    & \textcolor{red}{\textbf{9}} \\
\hline
Timing constraints (D2.6-02-066.02.02)  & 1    & 1    & 1    & 3    \\
\hline
Safety properties validation (D2.6-02-058.02)  & 3    & 3    & 3    & \textcolor{red}{\textbf{9}} \\
\hline
Logical properties assertion (D2.6-02-072)  &  3 & 3    & 3    & \textcolor{red}{\textbf{9}} \\
\hline
Check  of assertions (D2.6-02-072.1)  & 3    & 3    & 3    & \textcolor{red}{\textbf{9}} \\
\hline
\end{tabular}


\begin{author_comment}
Timing constraints like time-outs can be modelled.
\end{author_comment}


Does the language allow to  formalize (D2.6-02-069):

\begin{tabular}{|l | c | c | c | c|}
\hline
& \textcolor{green}{Author} & \textcolor{blue}{Assessor 1} & \textcolor{magenta}{Assessor 2} & Total \\
\hline 
State machines  & 3    & 3    & 3    & \textcolor{red}{\textbf{9}} \\
\hline
Time-outs  & 2    & 2    & 1    & \textcolor{blue}{5}  \\
\hline
Truth tables  & 3    & 3    & 3    & \textcolor{red}{\textbf{9}} \\
\hline
Arithmetic  & 3    & 3    & 3    & \textcolor{red}{\textbf{9}} \\
\hline
Braking curves  & 2    & 2    & 2    & \textcolor{blue}{6} \\
\hline
Logical statements & 3    & 3    & 3    & \textcolor{red}{\textbf{9}} \\
\hline
Message and fields & 3    & 3    & 3    & \textcolor{red}{\textbf{9}} \\
\hline
\end{tabular}


\begin{author_comment}
Braking curves have already been specifyed in classical b  in past industrial projects. However this need specific skills.
\end{author_comment}


\begin{assessor2}
The specification of braking curves is not easy in classical B as the time properties are a non trivial modeling aspects in classical B. Therefore these aspects require deep knowledge of B models.
\end{assessor2}

\paragraph{Additional comments on semi-formal  model} Do you think your semi-formal  model is sufficient to cover a safe design of the on-board unit until code generation ?
All comments on links to  other models, validation and verification activities are welcomed.



\begin{author_comment}
Classical B  approach is well adapted for software design untill code generation. However a formal approach as Event B \ref{chap:eventB} is more relevant to be used for system analysis and design.
\end{author_comment}


\subsection{Strictly formal model}

Concerning strictly formal model, how the WP2 requirements are covered ?

\begin{tabular}{|l | c | c | c | c|}
\hline
& \textcolor{green}{Author} & \textcolor{blue}{Assessor 1} & \textcolor{magenta}{Assessor 2} & Total \\
\hline 
Consistency to SFM (D2.6-02-049.2) & 3    & 3    & 3    & \textcolor{red}{\textbf{9}} \\
\hline
Coverage of SSRS (D2.6-02-049.2)  & 3    & 3    & 3    & \textcolor{red}{\textbf{9}} \\
\hline
Traceability to  SSRS (D2.6-02-049.3)  & 2    & 2    & 2    & \textcolor{blue}{6} \\
\hline
Extensible to software design (D2.6-02-051)  & 3    & 3    & 3    & \textcolor{red}{\textbf{9}} \\
\hline
Safety function isolation (D2.6-02-052)  & 3    & 3    & 3    & \textcolor{red}{\textbf{9}} \\
\hline 
Safety properties formalisation (D2.6-02-057)  & 2    & 3    & 2    & \textcolor{magenta}{7} \\
\hline
Logical expression (D2.6-02-066.02.01)  & 3    & 3    & 3    & \textcolor{red}{\textbf{9}} \\
\hline
Timing constraints (D2.6-02-066.02.02)  & 2    & 2    & 2    & \textcolor{blue}{6} \\
\hline
Safety properties validation (D2.6-02-058.03)  & 3    & 2    & 3    & \textcolor{magenta}{8} \\
\hline
Logical properties assertion (D2.6-02-072)  & 3    & 3    & 3    & \textcolor{red}{\textbf{9}} \\
\hline
Proof of assertions (D2.6-02-072.2)  & 3    & 3    & 3    & \textcolor{red}{\textbf{9}} \\
\hline
\end{tabular}



\begin{author_comment}
Timing constraints like time-outs can be modelled.
\end{author_comment}


\begin{assessor2}
It is correct, that timing aspects can be modelled in classical B, but this is not a main feature and requires more complicated commands.
\end{assessor2}

Does the language allow to  formalize (D2.6-02-070):

\begin{tabular}{|l | c | c | c | c|}
\hline
& \textcolor{green}{Author} & \textcolor{blue}{Assessor 1} & \textcolor{magenta}{Assessor 2} & Total \\
\hline 
State machines  & 3    & 3    & 3    & \textcolor{red}{\textbf{9}} \\
\hline
Time-outs  & 2    & 2    & 1    & \textcolor{blue}{5} \\
\hline
Truth tables  & 3    & 3    & 3    & \textcolor{red}{\textbf{9}} \\
\hline
Arithmetic  & 3    & 3    & 3    & \textcolor{red}{\textbf{9}} \\
\hline
Braking curves  & 2    & 2    & 2    & \textcolor{blue}{6} \\
\hline
Logical statements & 3    & 3    & 3     & \textcolor{red}{\textbf{9}} \\
\hline
Message and fields & 3    & 3    & 3    & \textcolor{red}{\textbf{9}} \\
\hline
\end{tabular}


\begin{author_comment}
Braking curves have already been specifyed in classical b  in past industrial projects. However this need specific skills.
\end{author_comment}


\paragraph{Additional comments on semi-formal  model} Do you think your strictly formal  model can be directly defined from the SSRS ?
All comments on links to  other models, validation and verification activities are welcomed.



\begin{author_comment}
Classical B  approach is well adapted for software design untill code generation. However a formal approach as Event B \ref{chap:eventB} is more relevant to be used for system analysis and design.
\end{author_comment}


\section{Software design}
This section discusses the usage of the approach for software design.
It can be skipped depending the results of \ref{main_usage}.

\subsection{Functional design}

How the approach allows to  produce a functional software model of the on-board unit ?

\begin{tabular}{|l | c | c | c | c|}
\hline
& \textcolor{green}{Author} & \textcolor{blue}{Assessor 1} & \textcolor{magenta}{Assessor 2} & Total \\
\hline
Derivation from system semi-formal model  & 3    & 3    & 3    & \textcolor{red}{\textbf{9}} \\
\hline 
Software architecture description  & 3    & 3    & 3    & \textcolor{red}{\textbf{9}} \\
\hline
Software constraints  & 3    & 3    & 3    & \textcolor{red}{\textbf{9}} \\
\hline
Traceability  & 3    & 3    & 3    & \textcolor{red}{\textbf{9}} \\
\hline
Executable  & 3     & 3    & 3    & \textcolor{red}{\textbf{9}} \\
\hline
\end{tabular}

\subsection{SSIL4 design}

How the approach allows to  produce in safety a software model ?

\begin{tabular}{|l | c | c | c | c|}
\hline
& \textcolor{green}{Author} & \textcolor{blue}{Assessor 1} & \textcolor{magenta}{Assessor 2} & Total \\
\hline
Derivation from system semi-formal or strictly formal model  & 3    & 3    & 3    & \textcolor{red}{\textbf{9}} \\
\hline 
Software architecture description  & 3    & 3    & 3    & \textcolor{red}{\textbf{9}} \\
\hline
Software constraints  & 3    & 3    & 3    & \textcolor{red}{\textbf{9}} \\
\hline
Traceability  & 3    & 3    & 3    & \textcolor{red}{\textbf{9}}  \\
\hline
Executable  & 3    & 3    & 3    & \textcolor{red}{\textbf{9}} \\
\hline
Conformance to EN50128 § 7.2  & 3    & 3    & 3    & \textcolor{red}{\textbf{9}} \\
\hline
Conformance to EN50128 § 7.3  & 3    & 3    & 3    & \textcolor{red}{\textbf{9}} \\
\hline
Conformance to EN50128 § 7.4  & 3    & 3    & 3    & \textcolor{red}{\textbf{9}} \\
\hline
\end{tabular}

Which criteria for software architecture are covered by the methodology
(see EN50128 table A.3) :

\begin{tabular}{|l | c | c | c | c|}
\hline
& \textcolor{green}{Author} & \textcolor{blue}{Assessor 1} & \textcolor{magenta}{Assessor 2} & Total \\
\hline
Defensive programming  & 3    & 3    & 3    & \textcolor{red}{\textbf{9}} \\
\hline 
Fault detection \& diagnostic  & 1    & 2    & 1    & 4    \\
\hline
Error detecting code  & 1    & 1    & \textcolor{green}{0} & 2    \\
\hline
Failure assertion programming & 2     & 2    & 2    & \textcolor{blue}{6} \\
\hline
Diverse programming & \textcolor{green}{0} & \textcolor{green}{0} & \textcolor{green}{0} & \textcolor{green}{0} \\
\hline
Memorising executed cases & \textcolor{green}{0} & \textcolor{green}{0} & \textcolor{green}{0} & \textcolor{green}{0} \\
\hline
Software error effect analysis & \textcolor{green}{0} & \textcolor{green}{0} & \textcolor{green}{0} & \textcolor{green}{0} \\
\hline
Fully defined interface & 3    & 3    & 3    & \textcolor{red}{\textbf{9}} \\
\hline
Modelling  & 3    & 3    & 3    & \textcolor{red}{\textbf{9}} \\
\hline
Structured methodology & 3    & 3    & 3    & \textcolor{red}{\textbf{9}} \\
\hline
\end{tabular}

\section{Software code generation}
This section discusses the usage of the approach for software code generation.
It can be skipped depending the results of \ref{main_usage}.

Which criteria for software design and implementation are covered by the methodology
(see EN50128 table A.4) :

\begin{tabular}{|l | c | c | c | c|}
\hline
& \textcolor{green}{Author} & \textcolor{blue}{Assessor 1} & \textcolor{magenta}{Assessor 2} & Total \\
\hline
Formal methods  & 3    & 3    & 3    & \textcolor{red}{\textbf{9}} \\
\hline 
Modeling  & 3    & 3    & 3    & \textcolor{red}{\textbf{9}} \\
\hline
Modular approach (mandatory) & 3    & 3    & 3    & \textcolor{red}{\textbf{9}} \\
\hline
Components & 3    & 3    & 3    & \textcolor{red}{\textbf{9}} \\
\hline
Design and coding standards (mandatory) & 3    & 3    & 3    & \textcolor{red}{\textbf{9}} \\
\hline
Strongly typed programming language & 3    & 3    & 3    & \textcolor{red}{\textbf{9}} \\
\hline

\end{tabular}



\section{Main usage of the tool}
\label{main_usage}

This section discusses the main usage of the tool.

Which task are covered by the tool ?


\begin{tabular}{|l | c | c | c | c|}
\hline
& \textcolor{green}{Author} & \textcolor{blue}{Assessor 1} & \textcolor{magenta}{Assessor 2} & Total \\
\hline 
Modelling support & 3    & 3    & 3    & \textcolor{red}{\textbf{9}} \\
\hline
Automatic translation  & 3    & 3    & 2    & \textcolor{magenta}{8} \\
\hline
Code Generation  & 3    & 3    & 3    & \textcolor{red}{\textbf{9}} \\
\hline
Model verification & 3    & 3    & 3     & \textcolor{red}{\textbf{9}} \\
\hline
Test generation & \textcolor{green}{0} & \textcolor{green}{0} & \textcolor{green}{0} & \textcolor{green}{0} \\
\hline
Simulation, execution, debugging & 2     & 2    & 2    & \textcolor{blue}{6} \\
\hline
Formal proof & 3    & 3    & 3    & \textcolor{red}{\textbf{9}} \\
\hline
\end{tabular}

\paragraph{Modelling support}
Does the tool provide a  textual or a graphical editor ?

\begin{author_comment}
Textual editor
\end{author_comment}

\paragraph{Automatic translation and code generation}
Which translation or code generation is supported by the tool ?

\begin{author_comment}
Automatic translation to  C, Ada or HIA is possible with existing tools. Automatic translators to other languages can be developed.
\end{author_comment}

\paragraph{Model verification}
Which verification on models are provided by the tool?

\begin{author_comment}
simulation, verification, validation and formal proof, test coverage
\end{author_comment}

\paragraph{Test generation}
Does the tool allow to generate tests ? For  which purpose ?

\begin{author_comment}
No
\end{author_comment}

\paragraph{Simulation, execution, debugging}
Does the tool allow to simulate or to debbug step by step a model or a code ?

\begin{author_comment}
Code can be simulated step  by step with specific tool of the target language or animated with ProB \url{http://www.tools.clearsy.com/wp1/?page_id=124}.
\end{author_comment}


\paragraph{Formal proof}
Does the tool allow formal proof ?  How ?

\begin{author_comment}
Yes, a set of rules describes how to  produce proof obligations to cover verification of the model (type verification, invariant preservation, refinement,...)
A set of rules can be defined to  write proofs.  
\end{author_comment}

\section{Use of the tool}


According WP2 requirements, give a note for characteristics of the use of the tool (from 0 to 3) :

\begin{tabular}{|l | c | c | c | c|}
\hline
& \textcolor{green}{Author} & \textcolor{blue}{Assessor 1} & \textcolor{magenta}{Assessor 2} & Total \\
\hline 
Open Source (D2.6-02-074) & 1    & 1    & 1    & 3    \\
\hline 
Portability to operating systems (D2.6-02-075) & 3    & 3    & 3    & \textcolor{red}{\textbf{9}} \\
\hline
Cooperation of tools (D2.6-02-076) & 2    & 2    & 2    & \textcolor{blue}{6} \\
\hline
Robustness (D2.6-02-078) & 3    & 3    & 3    & \textcolor{red}{\textbf{9}} \\
\hline
Modularity (D2.6-02-078.1) & 3    & 3    & 3    & \textcolor{red}{\textbf{9}} \\
\hline
Documentation management (D2.6-02-078.02) & 3    & 3    & \textcolor{green}{0} & \textcolor{blue}{6} \\
\hline
Distributed software development (D2.6-02-078.03)  & 2    & 3    & 2    & \textcolor{magenta}{7} \\
\hline
Simultaneous multi-users (D2.6-02-078.04)   & 2    & 3    & 2    & \textcolor{magenta}{7} \\
\hline
Issue tracking (D2.6-02-078.05) & \textcolor{green}{0} & \textcolor{green}{0} & \textcolor{green}{0} & \textcolor{green}{0} \\
\hline
Differences between models (D2.6-02-078.06) & 2    & 2    & 2    & \textcolor{blue}{6} \\
\hline
Version management (D2.6-02-078.07) & 1    & 1    & 1    & 3    \\
\hline
Concurrent version development (D2.6-02-078.08) & 1    & 1    & 1    & 3    \\
\hline
Model-based version control (D2.6-02-078.09) & \textcolor{green}{0} & \textcolor{green}{0} & \textcolor{green}{0} & \textcolor{green}{0} \\
\hline
Role traceability (D2.6-02-078.10) & 1    & 2    & 1    & 4    \\
\hline
Safety version traceability (D2.6-02-078.11) & \textcolor{green}{0} & \textcolor{green}{0} & \textcolor{green}{0} & \textcolor{green}{0} \\
\hline
Model traceability (D2.6-02-079) & 2    & 1    & 2    & \textcolor{blue}{5} \\
\hline
Tool chain integration & 2    & 2    & 2    & \textcolor{blue}{6} \\
\hline
Scalability & 3    & 3    & 3   & \textcolor{red}{\textbf{9}} \\
\hline
\end{tabular}

\begin{author_comment}
The tool is partly open-source, but it is free for use.
For more details \url{http://www.atelierb.eu/outil-atelier-b/}.
\end{author_comment}


\begin{assessor2}
Atelier B is not a tool for document management. It is a formal modeling and verification tool.
\end{assessor2}

\section{Certifiability}

This section discusses how the tool can be classified according EN50128 requirements (D2.6-02-085).


\begin{tabular}{|l | c | c | c | c|}
\hline
& \textcolor{green}{Author} & \textcolor{blue}{Assessor 1} & \textcolor{magenta}{Assessor 2} & Total \\
\hline 
Tool manual (D.2.6-01-42.02) & 3    & 3    & 3    & \textcolor{red}{\textbf{9}} \\
\hline
Proof of correctness (D.2.6-01-42.03)   & 3    & 3    & 3    & \textcolor{red}{\textbf{9}} \\
\hline
Existing industrial  usage  & 3     & 3    & 3    & \textcolor{red}{\textbf{9}} \\
\hline
Model verification & 3    & 3    & 3    & \textcolor{red}{\textbf{9}} \\
\hline
Test generation & \textcolor{green}{0} & \textcolor{green}{0} & \textcolor{green}{0} & \textcolor{green}{0} \\
\hline
Simulation, execution, debugging & 3    & 3    & 2    & \textcolor{magenta}{8} \\
\hline
Formal proof & 3    & 3    & 3    & \textcolor{red}{\textbf{9}} \\
\hline
\end{tabular}

\paragraph{Other elements for tool certification}


\begin{author_comment}
Method and tool have already been used to  develop numbers of  SIL4 software for railway.
\end{author_comment}


\section{Other comments}
Please to  give free comments on the approach.
