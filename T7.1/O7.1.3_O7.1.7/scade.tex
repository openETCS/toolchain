\chapter{SCADE}

\begin{description}
\item[\textcolor{green}{Author}] Author of the approaches description  Uwe Steinke (Siemens)
\item[\textcolor{blue}{Assessor 1}] First assessor of the approaches David Mentré (MERCE)
\item[\textcolor{magenta}{Assessor 2}] Second assessor of the approaches Cécile Braunstein (Uni. Bremen)
\end{description}

In the sequel, main text is under the responsibilities of the author.

\begin{author_comment}
Author can add comments using this format at any place.
\end{author_comment}

\begin{assessor1}
First assessor can add comments using this format at any place.
\end{assessor1}

\begin{assessor2}
Second assessor can add comments using this format at any place.
\end{assessor2}

When a note is required, please follow this list :
\begin{description}
\item[0] not recommended, not adapted, rejected
\item[1] weakly recommended, adapted after major improvements, weakly rejected
\item[2] recommended, adapted (with light improvements if necessary)  weakly accepted
\item[3] highly recommended, well adapted,strongly accepted
\item[*] difficult to evaluate with a note (please add a comment under the table)
\end{description}

All the notes can be commented under each table.

\section{Presentation}

This section gives a quick presentation of the approach and the tool.

\begin{description}
\item[Name:] SCADE Suite / SCADE System / SCADE LifeCycle
\item[Web site: ] \url{http://esterel-technologies.com }
\item[Licence: ] Commercial
\end{description}

\paragraph{Abstract} Short abstract on the approach and tool (10 lines max)

SCADE is a formal modelling language targeted for safety-critical embedded control applications
in the avionics, rail, automotive and industrial automation domain. SCADE source code can be
written as text (for anyone who likes writing plain text) or (more usual) as schematic diagrams.

SCADE models are synchronously clocked data flow and state machines, that can be nested
and intermixed with each other without limitations. SCADE provides DO-178B- and EN50128-
certified code generators producing C or ADA code as output. SCADE models are therefore concrete, deterministic, executable and verifiable; it allows the production of rapid prototype as
well as of safety related target system software.

SCADE comes with an integrated development environment (SCADE Suite IDE) including
code generator, graphical simulator, model checker/prover, model test coverage analyzer, report
generators, version and requirements management gateway with interfaces to various other tools
like static code and timing analyzers, System/SysML modelling tools etc.. The IDE provides
automatization interfaces to be controlled from external tools, and all SCADE tools itself can
also be used in batch mode. In addition, plugins for Eclipse integration are available.

The SCADE paradigm of synchronously clocked data flow and state machines works perfect for
embedded control or industry automation software. It is less suitable for tasks like text processing
or computer graphic applications. SCADE models do not only describe the structure of software;
instead, they are the software implementation itself too. System architectures typically require a
higher abstraction means of description at top level like SysML. While SysML modelling can
be achieved with any SysML tools, SCADE System provides an automatic transformation from
SysML to SCADE.

\paragraph{Publications} Short list of publications on the approach (5 max)

\begin{itemize}
	\item \url{http://www.interested-ip.eu/}
  \item \url{http://http://www.interested-ip.eu/final-report.html/}

\end{itemize}

\section{Main usage of the approach}
\label{main_usage}
This section discusses the main usage of the approach.

According to the figure \ref{fig:main_process}, for which phases do you recommend the approach (give a note from 0 to  3) :

\begin{tabular}{|l | c | c | c | c|}
\hline
& \textcolor{green}{Author} & \textcolor{blue}{Assessor 1} & \textcolor{magenta}{Assessor 2} & Total \\
\hline 
System Analysis & 1  & 1 &1 & 3 \\
\hline
Sub-system  formal  design &  3 & 3 &3 & 9\\
\hline
Software design & 3  & 3 & 3 & 9 \\
\hline
Software code generation & 3 & 3 &3 & 9 \\
\hline
\end{tabular}

\begin{author_comment}
SCADE can be used for analyzing tasks on system level, especially to clarify complex system behaviour and functions by practical modelling, execution, simulation and test. For a higher abstraction level, this should be enhanced with system modelling languages as SysML.
\end{author_comment}
According to the figure \ref{fig:main_process}, for which type of activities do you recommend the approach (give a note from 0 to  3) :

\begin{tabular}{|l | c | c | c | c|}
\hline
& \textcolor{green}{Author} & \textcolor{blue}{Assessor 1} & \textcolor{magenta}{Assessor 2} & Total \\
\hline 
Documentation &  3  & 2 &3 & 8  \\
\hline
Modeling &  3  & 3 &3 &  9 \\
\hline
Design &  3  & 3 &3 & 9 \\
\hline
Code generation &  3  & 3 &3 & 9 \\
\hline
Verification &  3  & 2 &3 & 8 \\
\hline
Validation &  3  & 2 &3 & 8 \\
\hline
Safety analyses &  0  & 1 &0 & 1 \\
\hline
\end{tabular}


\begin{assessor1}
Verification capabilities of SCADE Verifier are limited by state space
explosion. As far as I know, very few SCADE users are doing formal
verification except for very small designs.

The test facilities are however much used.
\end{assessor1}


\paragraph{Known usages} Have you some examples of usage of this approach to  compare with the OpenETCS objectives ?

Field of usage: Safety critical systems like
\begin{itemize}
	\item Rail interlocking systems
	\item Rail track vacancy detection systems
	\item Rail train control systems
	\item Rail Level-crossing protection systems
	\item Avionic flight controllers
\end{itemize}

\section{Language}
This section discusses the main element of the language.

Which are the main characteristics of the language :

\begin{tabular}{|l | c | c | c | c|}
\hline
& \textcolor{green}{Author} & \textcolor{blue}{Assessor 1} & \textcolor{magenta}{Assessor 2} & Total \\
\hline 
Informal language &  0  & 0 &0 &  0 \\
\hline 
Semi-formal language &  0  & 3 &0 &  3 \\
\hline
Formal language &  3  & 3 &3 &  9 \\
\hline
Structured language &  3  & 3 &3 & 9 \\
\hline
Modular language &  3  & 2 &3 & 8 \\
\hline
Textual language & 3 & 3 &3 & 9 \\
\hline
Mathematical symbols or code & 3 & 3 &3 & 9 \\
\hline
Graphical language & 3 & 3 &3 & 9 \\
\hline
\end{tabular}

According WP2 requirements, give a note for the capabilities of the language (from 0 to 3) :

\begin{tabular}{|l | c | c | c | c|}
\hline
& \textcolor{green}{Author} & \textcolor{blue}{Assessor 1} & \textcolor{magenta}{Assessor 2} & Total \\
\hline
Declarative formalization of properties (D.2.6-X-28) &
2  & 2 &2 & 6 \\
\hline
Simple formalization of properties (D.2.6-X-28.1) &
2 & 2 &2 & 6 \\
\hline
Scalability : capability to design large model &  3
& 2 &3 & 8 \\
\hline
Easily translatable to other languages (D.2.6-X-28) &
3  & 3 &3 & 9 \\
\hline
Executable directly (D.2.6-X-33) & 3  & 3 &3 & 9 \\
\hline
Executable after translation to a code (D.2.6-X-33) &
3& 3 &3 & 9 \\
(precise if the translation is automatic) &  3& 3 &3 & 9 \\
\hline
Simulation, animation (D.2.6-X-33) &  3 & 3 &3 & 9 \\
\hline
Easily understandable (D.2.6-X-27) &  3& 3 &3 & 9 \\
\hline
Expertise level needed (0 High level, 3 few level) &
2 & 2 &2 & 6 \\
\hline
Standardization (D.2.6-X-29) &  3& 0 &3 & 6 \\
\hline
Documented (D.2.6-X-29) &  3 & 2 &3 & 8 \\
\hline
Extensible language (D.2.6-01-28) &  2& 2 &1 & 5 \\
\hline
\end{tabular}
\begin{author_comment}
SCADE is a strictly textual and graphical formal language. It allows to be extended with user-defined operators. Especially the graphical representation is easy to learn and understand; nevertheless the rich tool suite covering most aspects of a EN50128 compliant process causes an appropriate learning effort by amount.
\end{author_comment}


\begin{assessor1}
  I am still skeptical that graphical representation is the best way
  to represent big systems.

  SCADE language lacks some facilities, like library parametrization
  by parameters (like OCaml functor, C++ template or Ada generics).
\end{assessor1}

\paragraph{Documentation} Describe how the language is documented, the existing guidelines, coding rules, standardization...

SCADE provides a detailed documentation set:

\begin{itemize}
	\item Getting Started
	\item SCADE Language Tutorial
	\item SCADE Suite User Manual
	\item SCADE Suite Technical Manual
	\item SCADE Suite Libraries Manual
	\item SCADE Language Primer
	\item SCADE Language Reference Manual
	\item Gateway Guidelines for LabView, Rhapsody, Simulink
	\item RTOS Adaptor Guidelines
	\item Timing and Stack Analysis Tools
	\item SCADE LifeCycle Documentation
	\item SCADE Suite Metamodels
	\item SCADE Glossary
	\item ...
\end{itemize}


\paragraph{Language usage} Describe the possible restriction on the language. 
SCADE is less suitable for tasks like text processing or computer graphic applications.


\section{System Analysis}
This section discusses the usage of the approach for system analysis.
It can be skipped depending the results of \ref{main_usage}.

Acoording WP2 requirements, how the approach can be involved for the sub-system requirement specification ?

\begin{tabular}{|l | c | c | c | c|}
\hline
& \textcolor{green}{Author} & \textcolor{blue}{Assessor 1} & \textcolor{magenta}{Assessor 2} & Total \\
\hline
Independent System functions definition (D.2.6-X-10.1.1)  &
3& 3 &3 &  9 \\
\hline 
System architecture design (D.2.6-X-10.1.2) & 1 & 1 &
0& 3  \\
\hline
System data flow identification (D.2.6-X-10.2.3)  &
3& 2 &3 &  8 \\
\hline
Sub-system focus (D.2.6-X-10.2.4)  &  2& 2 &2 &  6 \\
\hline
System interfaces definition (D.2.6-X-10.2.5)  &
2& 3 &3 & 8  \\
\hline
System requirement allocation (D.2.6-X-10.3)  &  3&
1 &3 &  7 \\
\hline
Traceability with SRS (D.2.6-X-10.5)  &  3& 3 &3 & 9  \\
\hline
Traceability with Safety activities (D.2.6-X-11)  &
3 & 3 &3 & 9  \\
\hline
\end{tabular}

\begin{author_comment}
Although SCADE is not made for system analysis, it can be used for the following aspects on system level: Modelling of (separate) system functions, data flows, state machines and interfaces. It provides an excellent tracebility support between many different kinds of documents and other tools.  
\end{author_comment}


\section{Sub-System formal design}
This section discusses the usage of the approach for sub-system formal design.
It can be skipped depending the results of \ref{main_usage}.

Two kinds of model can be planned during this phase: semi-formal models to  cover the SSRS (D.2.6-X-12.1) and strictly formal  models to  focuss on some functional and safety aspects (D.2.6-X-14).  Obviously some strictly  formal means can be used to define the semi-formal  model.

\subsection{Semi-formal model}

\begin{author_comment}
SCADE models are formal. Since the following table addresses many aspects that SCADE covers in a formal way it is filled anyway. But keep in mind: it's formal - instead of semi-formal.  
\end{author_comment}


Concerning formal model, how the WP2 requirements are covered ?

\begin{tabular}{|l | c | c | c | c|}
\hline
& \textcolor{green}{Author} & \textcolor{blue}{Assessor 1} & \textcolor{magenta}{Assessor 2} & Total \\
\hline 
Consistency to SSRS (D.2.6-X-12.2) & 2 & 2 & & - \\
\hline
Coverage of SSRS (D.2.6-X-12.2.1)  & 3 & 2 & & - \\
\hline
Coverage of SSHA (D.2.6-X-12.2.2)  &  3& 3 & & - \\
\hline
Management of requirement justification (D.2.6-X-12.2.3)  &
3& 3 & & - \\
\hline
Traceability to  SSRS (D.2.6-X-12.2.5)  &  3& 3 & & - \\
\hline
Traceability of exported requirements (D.2.6-X-12.2.6)  &
3& 3 & & - \\
\hline
Simulation or animation (D.2.6-X-13 partial)  &  3&
3& & - \\
\hline
Execution (D.2.6-X-13 partial)  & 3 & 3 & & - \\
\hline
Extensible to strictly formal model (D.2.6-X-14.3) &
3  & 3 & & - \\
\hline
Easy to  refine towards strictly formal model (D.2.6-X-14.4) &
3 & 3 & & - \\
\hline
Extensible and modular design (D.2.6-X-15)  & 3 & 2 & & - \\
\hline
Extensible to software architecture and design (D.2.6-X-30)   &
3& 3 & & - \\
\hline
\end{tabular}

Concerning safety properties management, how the WP2 requirements are covered ?

\begin{tabular}{|l | c | c | c | c|}
\hline
& \textcolor{green}{Author} & \textcolor{blue}{Assessor 1} & \textcolor{magenta}{Assessor 2} & Total \\
\hline 
Safety function isolation (D.2.6-X-17)  &  2& 2 & & - \\
\hline 
Safety properties formalisation (D.2.6-X-22)  &  2&
1 & & - \\
\hline
Logical expression (D.2.6-X-28.2.2)  &  3& 3 & & - \\
\hline
Timing constraints (D.2.6-X-28.2.3)  &  2 & 2 & & - \\
\hline
Safety properties validation (D.2.6-X-23.2)  &  2& 2 & & - \\
\hline
Logical properties assertion (D.2.6-X-34)  &  2& 2 & & - \\
\hline
Check  of assertions (D.2.6-X-34.1)  &  2& 2 & & - \\
\hline
\end{tabular}

\begin{author_comment}
SCADE is a modelling language for functions. Therefore, only the functional aspects of properties are addressed.  
\end{author_comment}


\begin{assessor1}
  SCADE lacks facilities to check for example data flow properties
  (inputs, outputs, in-outs) of a function.
\end{assessor1}


Does the language allow to  formalize (D.2.6-X-31):

\begin{tabular}{|l | c | c | c | c|}
\hline
& \textcolor{green}{Author} & \textcolor{blue}{Assessor 1} & \textcolor{magenta}{Assessor 2} & Total \\
\hline 
State machines  & 3 & 3 & & - \\
\hline
Time-outs  & 3 & 3 & & - \\
\hline
Truth tables  & 3 & 3 & & - \\
\hline
Arithmetic  & 3 & 3 & & - \\
\hline
Braking curves  & 3 & 3 & & - \\
\hline
Logical statements & 3 & 3 & & - \\
\hline
Message and fields & 2 & 2 & & - \\
\hline
\end{tabular}

\paragraph{Additional comments on semi-formal  model} Do you think your semi-formal  model is sufficient to cover a safe design of the on-board unit until code generation ?
All comments on links to  other models, validation and verification activities are welcomed.

\begin{author_comment}
SCADE is targeted for these purposes.   
\end{author_comment}


\subsection{Strictly formal model}

Concerning strictly formal model, how the WP2 requirements are covered ?

\begin{tabular}{|l | c | c | c | c|}
\hline
& \textcolor{green}{Author} & \textcolor{blue}{Assessor 1} & \textcolor{magenta}{Assessor 2} & Total \\
\hline 
Consistency to SFM (D.2.6-X-14.2) & 3 & 3 &3 & 9 \\
\hline
Coverage of SSRS (D.2.6-X-14.2)  &3 & 3 &3 &  9 \\
\hline
Traceability to  SSRS (D.2.6-X-14.3)  &3 & 3 &3 &  9 \\
\hline
Extensible to software design (D.2.6-X-16)  &3 & 3 &3 & 9  \\
\hline
Safety function isolation (D.2.6-X-17)  &  3& 2 &3 & 8  \\
\hline 
Safety properties formalisation (D.2.6-X-22)  &  2&
2&2 & 6  \\
\hline
Logical expression (D.2.6-X-28.2.2)  &  3&3 &3 & 9  \\
\hline
Timing constraints (D.2.6-X-28.2.3)  &  3&2 &3 & 8 \\
\hline
Safety properties validation (D.2.6-X-23.3)  &  2& 2 &2 & 6  \\
\hline
Logical properties assertion (D.2.6-X-34)  &  2& 2 &2 & 6 \\
\hline
Proof of assertions (D.2.6-X-34.2)  &  2 & 2 &3 & 7 \\
\hline
\end{tabular}

Does the language allow to  formalize (D.2.6-X-32):

\begin{tabular}{|l | c | c | c | c|}
\hline
& \textcolor{green}{Author} & \textcolor{blue}{Assessor 1} & \textcolor{magenta}{Assessor 2} & Total \\
\hline 
State machines  & 3 & 3 &3 & 9 \\
\hline
Time-outs  & 3 & 2 &3 & 8 \\
\hline
Truth tables  & 3 & 3 &3 & 9 \\
\hline
Arithmetic  & 3& 3 &3 & 9 \\
\hline
Braking curves  & 3& 3 &3 & 9 \\
\hline
Logical statements & 3& 3 &3 & 9 \\
\hline
Message and fields &3 & 2 &3 & 8 \\
\hline
\end{tabular}

\paragraph{Additional comments on semi-formal  model} Do you think your strictly formal  model can be directly defined from the SSRS ?
All comments on links to  other models, validation and verification activities are welcomed.

\begin{author_comment}
SCADE is targeted for these purposes.   
\end{author_comment}
\begin{assessor2}
A SCADE model my be directly defined from the SSRS, this implies that
code may directly be generated from the SSRS.
\end{assessor2}


\section{Software design}
This section discusses the usage of the approach for software design.
It can be skipped depending the results of \ref{main_usage}.

\subsection{Functional design}

How the approach allows to  produce a functional software model of the on-board unit ?

\begin{tabular}{|l | c | c | c | c|}
\hline
& \textcolor{green}{Author} & \textcolor{blue}{Assessor 1} & \textcolor{magenta}{Assessor 2} & Total \\
\hline
Derivation from system semi-formal model  & 3 & 3 &3 & 9 \\
\hline 
Software architecture description  & 3 & 3 &3 & 9 \\
\hline
Software constraints  & 3 & 2 &3 & 8 \\
\hline
Traceability  & 3 & 3 &3 & 9 \\
\hline
Executable  & 3 & 3 &3 & 9 \\
\hline
\end{tabular}

\subsection{SSIL4 design}

How the approach allows to  produce in safety a software model ?

\begin{tabular}{|l | c | c | c | c|}
\hline
& \textcolor{green}{Author} & \textcolor{blue}{Assessor 1} & \textcolor{magenta}{Assessor 2} & Total \\
\hline
Derivation from system semi-formal or strictly formal model  &
3 & 3 &3 & 9 \\
\hline 
Software architecture description  & 3 & 3 &3 & 9 \\
\hline
Software constraints  & 3 & 3 &3 & 9 \\
\hline
Traceability  & 3 & 3 &3 & 9 \\
\hline
Executable  & 3 & 3 &3 & 9 \\
\hline
Conformance to EN50128 § 7.2  & 3 & 3 &3 & 9 \\
\hline
Conformance to EN50128 § 7.3  & 3 & 3 &3 & 9 \\
\hline
Conformance to EN50128 § 7.4  & 3& 3 &3 & 9 \\
\hline
\end{tabular}

Which criteria for software architecture are covered by the methodology
(see EN50128 table A.3) :

\begin{tabular}{|l | c | c | c | c|}
\hline
& \textcolor{green}{Author} & \textcolor{blue}{Assessor 1} & \textcolor{magenta}{Assessor 2} & Total \\
\hline
Defensive programming  & 3* & 0 &3 &  6 \\
\hline 
Fault detection \& diagnostic  & 0* & 0 &0 & 0 \\
\hline
Error detecting code  & 0* & 0 &0 & 0 \\
\hline
Failure assertion programming & 1 & 0 &0 & 1 \\
\hline
Diverse programming & 0* & 0 &0 & 0 \\
\hline
Memorising executed cases & 0* & 1 &2 & 3 \\
\hline
Software error effect analysis & 0* & 0 &0 & 0 \\
\hline
Fully defined interface & 3 & 3 &3 & 9 \\
\hline
Modelling  & 3 & 3 &3 & 9 \\
\hline
Structured methodology & 3 & 3 &3 & 9 \\
\hline
\end{tabular}

\begin{author_comment}
The * are out of scope of SCADE. It does not support or prohibit these techniques in a specific way.   
\end{author_comment}


\section{Software code generation}
This section discusses the usage of the approach for software code generation.
It can be skipped depending the results of \ref{main_usage}.

Which criteria for software design and implementation are covered by the methodology
(see EN50128 table A.4) :

\begin{tabular}{|l | c | c | c | c|}
\hline
& \textcolor{green}{Author} & \textcolor{blue}{Assessor 1} & \textcolor{magenta}{Assessor 2} & Total \\
\hline
Formal methods  & 3 & 2 &3 & 8 \\
\hline 
Modeling  & 3 & 3 &3 & 9 \\
\hline
Modular approach (mandatory) & 3 & 3 &3 & 9 \\
\hline
Components & 3 & 3 &3 & 9 \\
\hline
Design and coding standards (mandatory) & 3 & 2 &3 & 8 \\
\hline
Strongly typed programming language & 3 & 3 &3 & 9 \\
\hline
\end{tabular}

\begin{assessor1}
  Does SCADE allow to check coding rules at diagram level? I don't
  think so.

  Set of properties and extent of their verification is limited in
  SCADE.
\end{assessor1}



\section{Main usage of the tool}
\label{main_usage}

This section discusses the main usage of the tool.

Which task are covered by the tool ?


\begin{tabular}{|l | c | c | c | c|}
\hline
& \textcolor{green}{Author} & \textcolor{blue}{Assessor 1} & \textcolor{magenta}{Assessor 2} & Total \\
\hline 
Modelling support & 3 & 3 &3 & 9 \\
\hline
Automatic translation  & 3 & 3 &3 & 9 \\
\hline
Code Generation  & 3 & 3 &3 & 9 \\
\hline
Model verification & 3 & 2 &2 & 7 \\
\hline
Test generation & 2 & 1 &1 & 4 \\
\hline
Simulation, execution, debugging & 3 & 3 &3 & 9 \\
\hline
Formal proof & 3 & 2 &2 & 7 \\
\hline
\end{tabular}

\paragraph{Modelling support}
Does the tool provide a  textual or a graphical editor ? 

\begin{itemize}
	\item Yes, both.
\end{itemize}


\paragraph{Automatic translation and code generation}
Which translation or code generation is supported by the tool ?

\begin{itemize}
	\item Validated translation of SCADE models into C or ADA code.
\end{itemize}

\paragraph{Model verification}
Which verification on models are provided by the tool?

\begin{itemize}
	\item Simulation
	\item animation
	\item verification via manual or script-based test suites
	\item Model test coverage for structural model coverage measurement
	\item Formal proving
\end{itemize}

\paragraph{Test generation}
Does the tool allow to generate tests ? For  which purpose ?

\begin{itemize}
	\item SCADE offers test suites to be built via test scripts
	\item For automatic model based test case generation tools like RT-Tester are applicable
\end{itemize}

\paragraph{Simulation, execution, debugging}
Does the tool allow to simulate or to debug step by step a model or a code ?

\begin{itemize}
	\item It allows simulation on a clock by clock base by executing the generated code while the model behaviour is visualized graphically
	\item Graphical model debugging with breakpoint capabilities
	\item Playback function for logfiles from the field
\end{itemize}

\paragraph{Formal proof}
Does the tool allow formal proof ?  How ?

\begin{itemize}
	\item SCADE integrates the Prover design verifier. 
	\item the provable properties have to be modelled with SCADE Suite and connected to the target model in an target-observer configuration
\end{itemize}


\begin{assessor1}
  Formal verification is limited by state space explosion.
\end{assessor1}

\section{Use of the tool}


According WP2 requirements, give a note for characteristics of the use of the tool (from 0 to 3) :

\begin{tabular}{|l | c | c | c | c|}
\hline
& \textcolor{green}{Author} & \textcolor{blue}{Assessor 1} & \textcolor{magenta}{Assessor 2} & Total \\
\hline 
Open Source (D2.6-X-36) &  0& 0 &0 & 0 \\
\hline 
Portability to operating systems (D2.6-X-37) &  3*&
1 &0 & 4 \\
\hline
Cooperation of tools (D2.6-X-38) &  3& 3 &2 & 8 \\
\hline
Robustness (D2.6-X-41) &  3& 3 &3 & 9 \\
\hline
Modularity (D2.6-X-41.1) &  3& 3 &3 & 9 \\
\hline
Documentation management (D.2.6-X-41.2) &  3& 3 &3 & 9 \\
\hline
Distributed software development (D.2.6-X-41.3)  &
2& 2 &2 & 6 \\
\hline
Simultaneous multi-users (D.2.6-X-41.4)   &  1& 1 &1 & 3 \\
\hline
Issue tracking (D.2.6-X-41.5) &  0&0 &0 & 0 \\
\hline
Differences between models (D.2.6-X-41.6) &  3& 3 &3 & 9 \\
\hline
Version management (D.2.6-X-41.7) &  3& 3 &3 & 9 \\
\hline
Concurrent version development (D.2.6-X-41.8) &  2&
2& * & 4* \\
\hline
Model-based version control (D.2.6-X-41.9) &  3& 3 &3 & 9 \\
\hline
Role traceability (D.2.6-X-41.10) &  0& 0 &0 & 0 \\
\hline
Safety version traceability (D.2.6-X-41.11) &  0& 0 &0 & 0 \\
\hline
Model traceability (D.2.6-01-035) & 3 & 3 &3 & 9 \\
\hline
Tool chain integration & 3 & 3 &* & 6* \\
\hline
Scalability & 3 & 3 &3 & 9 \\
\hline
\end{tabular}

\begin{author_comment}
* The SCADE tool suite requires MS Windows. The generated executable code is able to run on any operating system and on platforms without an operation system.   
\end{author_comment}

\begin{assessor2}managment
SCADE is a almost complete tool chain. Missing multi-users and system design.
\end{assessor2}
\section{Certifiability}

This section discusses how the tool can be classified according EN50128 requirements (D.2.6-X-50).

\begin{author_comment}
SCADE is targeted to be certifiable. Validation documentation is available.   
\end{author_comment}


\begin{tabular}{|l | c | c | c | c|}
\hline
& \textcolor{green}{Author} & \textcolor{blue}{Assessor 1} & \textcolor{magenta}{Assessor 2} & Total \\
\hline 
Tool manual (D.2.6-01-42.02) & 3 & 3 &3 & 9 \\
\hline
Proof of correctness (D.2.6-01-42.03)   & 3 & 3 &3 & 9 \\
\hline
Existing industrial  usage  & 3 & 3 &3 & 9 \\
\hline
Model verification & 3 & 2 &3 & 8 \\
\hline
Test generation & 2 & 2 &2 & 6 \\
\hline
Simulation, execution, debugging & 3 & 3 &3 & 9 \\
\hline
Formal proof & 3 & 2 &2 & 7 \\
\hline
\end{tabular}

\paragraph{Other elements for tool certification}

\section{Other comments}
Please to  give free comments on the approach.




% LocalWords:  SCADE
