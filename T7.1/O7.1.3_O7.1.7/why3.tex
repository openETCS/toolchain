\chapter{Why3}

\begin{description}
\item[\textcolor{green}{Author}] Author of the approaches description: David Mentré (Mitsubishi Electric R\&D Centre
Europe)
\end{description}


\section{Presentation}

This section gives a quick presentation of the approach and the tool.

\begin{description}
\item[Name:] Why3
\item[Web site: ] \url{http://why3.lri.fr/}
\item[Licence: ] GNU LGPL
\end{description}

\paragraph{Abstract} Short abstract on the approach and tool (10 lines max)

Why3 is a platform for deductive program verification. It provides a
rich language for specification and programming, called WhyML, and
relies on external theorem provers, both automated and interactive, to
discharge verification conditions. Why3 comes with a standard library
of logical theories (integer and real arithmetic, Boolean operations,
sets and maps, etc.) and basic programming data structures (arrays,
queues, hash tables, etc.). A user can write WhyML programs directly
and get correct-by-construction OCaml programs through an automated
extraction mechanism. WhyML is also used as an intermediate language
for the verification of C, Java, or Ada programs.



\paragraph{Publications} Short list of publications on the approach (5 max)

\begin{itemize}
	\item Why3: Shepherd Your Herd of Provers (BOOGIE 2011)
\url{http://proval.lri.fr/submissions/boogie11.pdf}

\item Verifying Two Lines of C with Why3: an Exercise in Program Verification (VSTTE 2012)
\url{http://why3.lri.fr/queens/queens.pdf}

\item Why3 -- Where Programs Meet Provers  (ESOP 2013)
\url{http://hal.inria.fr/hal-00789533}

\end{itemize}

\section{Evaluation}

The evaluation of this approach has been stop by the author before the end of the benchmark activity:


\begin{author_comment}

Main reason to stop Why3 model: Why3 and GNATprove have roughly the same capabilities but GNATprove is less error prone.

Both GNATprove and Why3 are using a contract approach, with pre and post-conditions on functions (and some kind of data invariant). Both have same expression capabilities with complicated data structures: sum types (record with discriminant in Ada), array, record, etc. GNATprove code can be compiled with an Ada compiler. Why3 can be translated to OCaml (at least for the latest git version). Both are using several automated SMT solvers, thus they are rather easy to use (once correct invariant are written ;-) ).

However GNATprove or at least Ada supports tasking, thus making concurrent models can be considered. This is not possible with Why3.

Moreover, Why3 heavily relies on axiomatizations that are fragile: it is easy to make a mistake when writing axioms. GNATprove has a fixed set of expression capabilities within its logical framework, but at least it is much less risky for a regular user which is not expert in formal methods.

\end{author_comment}



