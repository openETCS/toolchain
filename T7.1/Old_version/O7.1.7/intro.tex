% Start here


\chapter{Introduction}


The aim of this document is to report the results of the evaluation of the tools to include in the OpenETCS toolchain.

This evaluation task is part on work package WP7, task 1  "Primary tool Chain analyses and recommendations". According to the results of WP2, especially the OpenETCS process and the requirements on tools, the aim of this task is to determine the best candidates to  produce models of the on-board units, following the OpenETCS process and the associated tools.

Tools evaluation is linked to the means evaluation described in O7.1.3 " Evaluation o the models against WP2 requirements".


The second section of this document provides a template to describe the tools and a list of criteria according WP2 requirements on tools. The objectives of this description and criteria are to allow to determine the best means of description and associated.

The third section sums up the results of the evaluation at the end of the benchmark activities.

In Appendix, a section is dedicated to each tools evaluated during the benchmark activities :
\begin{itemize}
\item  CORE
\item  GOPRR
\item  ERTMSFormalSpecs
\item  SysML with Papyrus
\item  SysML with Entreprise Architect
\item  SCADE
\item  EventB 
\item  Classical B 
\item  Petri Nets
\item  System C
\item  GNATprove
\end{itemize}

For each tool, the initial  author of the evaluation is the partner in charge of the modelling. Two assessors, for each tool,  are in charge of the review of the evaluation and can correct it or add comments.

Languages and tool platforms are not covered by this document but in other output of WP7 : O7.1.3  "Evaluation of the means against the WP2 requirements" and O7.1.9 "Evaluation of each tool platform against WP2 requirements, independent of target tools".
Besides, Task 7.1 is focussing on design activities : despite that some tools can provide verification artefacts for example,  tools and means for validation, verification, test generation are in the scope of task 2 and will be analysed later.