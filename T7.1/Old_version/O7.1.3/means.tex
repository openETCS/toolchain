\chapter{Templates}

\begin{description}
\item[\textcolor{green}{Author}] Author of the approaches description  \todo{Name -  Company}
\item[\textcolor{blue}{Assessor 1}] First assessor of the approaches \todo{Name - Company}
\item[\textcolor{magenta}{Assessor 2}] Second assessor of the approaches \todo{Name - Company}
\end{description}

In the sequel, main text is under the responsibilities of the author.

\begin{author_comment}
Author can add comments using this format
\end{author_comment}

\begin{assessor1}
First assessor can add comments using this format
\end{assessor1}

\begin{assessor2}
Second assessor can add comments using this format
\end{assessor2}

When a note is required, please follow this list :
\begin{description}
\item[0] not recommended, not adapted, rejected
\item[1] weakly recommended, adapted after major improvements, weakly rejected
\item[2] recommended, adapted (with light improvements if necessary)  weakly accepted
\item[3] highly recommended, well adapted,strongly accepted
\item[*] difficult to evaluate with a note (please add a comment under the table)
\end{description}

All the notes can be commented under each table.

\section{Presentation}

This section gives a quick presentation of the approach.

\begin{description}
\item[Name] Name of the approach
\item[Web site] if available, how to  find information
\end{description}

\paragraph{Abstract} Short abstract on the approach (5 lines max)

\paragraph{Publications} Short list of publications on the approach (5 max)


\section{Main usage of the approach}
\label{main_usage}
This section discusses the main usage of the approach.

According to the figure \ref{fig:main_process}, for which phases do you recommend the approach (give a note from 0 to  3) :

\begin{tabular}{|l | c | c | c | c|}
\hline
& \textcolor{green}{Author} & \textcolor{blue}{Assessor 1} & \textcolor{magenta}{Assessor 2} & Total \\
\hline 
System Analysis & & & &  \\
\hline
Sub-system  formal  design & & & & \\
\hline
Software design & & & & \\
\hline
Software code generation & & & & \\
\hline
\end{tabular}

According to the figure \ref{fig:main_process}, for which type of activities do you recommend the approach (give a note from 0 to  3) :

\begin{tabular}{|l | c | c | c | c|}
\hline
& \textcolor{green}{Author} & \textcolor{blue}{Assessor 1} & \textcolor{magenta}{Assessor 2} & Total \\
\hline 
Documentation & & & &  \\
\hline
Modeling & & & &  \\
\hline
Design & & & & \\
\hline
Code generation & & & & \\
\hline
Verification & & & & \\
\hline
Validation & & & & \\
\hline
Safety analyses & & & & \\
\hline
\end{tabular}

\paragraph{Known usages} Have you some examples of usage of this approach to  compare with the OpenETCS objectives ?

\section{Language}
This section discusses the main element of the language.

According WP2 requirements, give a note for the characteristics of the language (from 0 to 3) :

\begin{tabular}{|l | c | c | c | c|}
\hline
& \textcolor{green}{Author} & \textcolor{blue}{Assessor 1} & \textcolor{magenta}{Assessor 2} & Total \\
\hline 
Informal language & & & &  \\
\hline 
Semi-formal language & & & &  \\
\hline
Formal language & & & &  \\
\hline
Structured language & & & & \\
\hline
Modular language & & & & \\
\hline
Extensible language (D.2.6-01-28) & & & & \\
\hline
Textual language & & & & \\
\hline
Mathematical symbols or code & & & & \\
\hline
Graphical language & & & & \\
\hline
Declarative and simple formalization of properties (D.2.6-X-27) & & & & \\
\hline
Scalability : capability to design large model & & & & \\
\hline
Easily translatable to other languages (D.2.6-X-28) & & & & \\
\hline
Executable & & & & \\
\hline
Simulation, animation & & & & \\
\hline
Easily understandable (D.2.6-X-26) & & & & \\
\hline
Expertise level needed (0 High level, 3 few level) & & & & \\
\hline
Standardization & & & & \\
\hline
\end{tabular}


\paragraph{Documentation} Describe how the language is documented, the existing guidelines, coding rules, standardization...

\paragraph{Language usage} Describe the possible restriction on the language

\section{System Analysis}
This section discusses the usage of the approach for system analysis.
It can be skipped depending the results of \ref{main_usage}.

Acoording WP2 requirements, how the approach can be involved for the sub-system requirement specification ?

\begin{tabular}{|l | c | c | c | c|}
\hline
& \textcolor{green}{Author} & \textcolor{blue}{Assessor 1} & \textcolor{magenta}{Assessor 2} & Total \\
\hline
Independent System functions definition (D.2.6-X-10.1.1)  & & & &  \\
\hline 
System architecture design (D.2.6-X-10.1.2) & & & &  \\
\hline
System data flow identification (D.2.6-X-10.1.3)  & & & &  \\
\hline
Sub-system focus (D.2.6-X-10.1.4)  & & & &  \\
\hline
System interfaces definition (D.2.6-X-10.1.5)  & & & &  \\
\hline
System requirement allocation (D.2.6-X-10.2)  & & & &  \\
\hline
Traceability with SRS (D.2.6-X-10.3)  & & & &  \\
\hline
Traceability with Safety activities (D.2.6-X-11)  & & & &  \\
\hline
\end{tabular}



\section{Sub-System formal design}
This section discusses the usage of the approach for sub-system formal design.
It can be skipped depending the results of \ref{main_usage}.


\subsection{Semi-formal model}

Concerning semi-formal model, how the WP2 requirements are covered ?

\begin{tabular}{|l | c | c | c | c|}
\hline
& \textcolor{green}{Author} & \textcolor{blue}{Assessor 1} & \textcolor{magenta}{Assessor 2} & Total \\
\hline 
Consistency to SSRS (D.2.6-X-12.2) & & & &  \\
\hline
Coverage of SSRS (D.2.6-X-12.2.1)  & & & &  \\
\hline
Traceability to  SSRS (D.2.6-X-12.3)  & & & &  \\
\hline
Simulation or animation (D.2.6-X-13 partial)  & & & &  \\
\hline
Execution (D.2.6-X-13 partial)  & & & &  \\
\hline
Extensible to strictly formal model (D.2.6-X-14.3) & & & &  \\
\hline
Easy to  refine towards strictly formal model (D.2.6-X-14.4) & & & &  \\
\hline
Extensible and modular design (D.2.6-X-15)  & & & &  \\
\hline
Extensible to software design (???)  & & & &  \\
\hline
Safety properties formalisation (D.2.6-01-20)  & & & &  \\
\hline
Safety properties validation (D.2.6-X-22 partial)  & & & &  \\
\hline
Logical properties assertion (D.2.6-X-32)  & & & &  \\
\hline
Check  of assertions (D.2.6-X-32.1)  & & & &  \\
\hline
\end{tabular}

Does the language allow to  formalize (D.2.6-X-29):

\begin{tabular}{|l | c | c | c | c|}
\hline
& \textcolor{green}{Author} & \textcolor{blue}{Assessor 1} & \textcolor{magenta}{Assessor 2} & Total \\
\hline 
State machines  & & & &  \\
\hline
Time-outs  & & & &  \\
\hline
Truth tables  & & & &  \\
\hline
Arithmetic  & & & &  \\
\hline
Braking curves  & & & &  \\
\hline
Logical statements & & & &  \\
\hline
Message and fields & & & &  \\
\hline
\end{tabular}

\paragraph{Additional comments on semi-formal  model} Do you think your semi-formal  model is sufficient to cover a safe design of the on-board unit until code generation ?
All comments on links to  other models, validation and verification activities are welcomed.

\subsection{Strictly formal model}

Concerning strictly formal model, how the WP2 requirements are covered ?

\begin{tabular}{|l | c | c | c | c|}
\hline
& \textcolor{green}{Author} & \textcolor{blue}{Assessor 1} & \textcolor{magenta}{Assessor 2} & Total \\
\hline 
Consistency to SFM (D.2.6-X-14.2) & & & &  \\
\hline
Coverage of SSRS (D.2.6-X-14.2)  & & & &  \\
\hline
Traceability to  SSRS (D.2.6-X-14.3)  & & & &  \\
\hline
Extensible to software design (D.2.6-X-16)  & & & &  \\
\hline
Safety properties formalisation (D.2.6-01-20)  & & & &  \\
\hline
Safety properties validation (D.2.6-X-22 partial)  & & & &  \\
\hline
Logical properties assertion (D.2.6-X-32)  & & & &  \\
\hline
Check  of assertions (D.2.6-X-32.2)  & & & &  \\
\hline
\end{tabular}

Does the language allow to  formalize (D.2.6-X-30):

\begin{tabular}{|l | c | c | c | c|}
\hline
& \textcolor{green}{Author} & \textcolor{blue}{Assessor 1} & \textcolor{magenta}{Assessor 2} & Total \\
\hline 
State machines  & & & &  \\
\hline
Time-outs  & & & &  \\
\hline
Truth tables  & & & &  \\
\hline
Arithmetic  & & & &  \\
\hline
Braking curves  & & & &  \\
\hline
Logical statements & & & &  \\
\hline
Message and fields & & & &  \\
\hline
\end{tabular}

\paragraph{Additional comments on semi-formal  model} Do you think your strictly formal  model can be directly defined from the SSRS ?
All comments on links to  other models, validation and verification activities are welcomed.


\section{Software design}
This section discusses the usage of the approach for software design.
It can be skipped depending the results of \ref{main_usage}.

\subsection{Functional design}

How the approach allows to  produce a functional software model of the on-board unit ?

\begin{tabular}{|l | c | c | c | c|}
\hline
& \textcolor{green}{Author} & \textcolor{blue}{Assessor 1} & \textcolor{magenta}{Assessor 2} & Total \\
\hline
Derivation from system semi-formal model  & & & &  \\
\hline 
Software architecture description  & & & &  \\
\hline
Software constraints  & & & &  \\
\hline
Traceability  & & & &  \\
\hline
Executable  & & & &  \\
\hline
\end{tabular}

\subsection{SSIL4 design}

How the approach allows to  produce in safety a software model ?

\begin{tabular}{|l | c | c | c | c|}
\hline
& \textcolor{green}{Author} & \textcolor{blue}{Assessor 1} & \textcolor{magenta}{Assessor 2} & Total \\
\hline
Derivation from system semi-formal or strictly formal model  & & & &  \\
\hline 
Software architecture description  & & & &  \\
\hline
Software constraints  & & & &  \\
\hline
Traceability  & & & &  \\
\hline
Executable  & & & &  \\
\hline
Conformance to EN50128 § 7.2  & & & &  \\
\hline
Conformance to EN50128 § 7.3  & & & &  \\
\hline
Conformance to EN50128 § 7.4  & & & &  \\
\hline
\end{tabular}

Which criteria for software architecture are covered by the methodology
(see EN50128 table A.3) :

\begin{tabular}{|l | c | c | c | c|}
\hline
& \textcolor{green}{Author} & \textcolor{blue}{Assessor 1} & \textcolor{magenta}{Assessor 2} & Total \\
\hline
Defensive programming  & & & &  \\
\hline 
Fault detection \& diagnostic  & & & &  \\
\hline
Error detecting code  & & & &  \\
\hline
Failure assertion programming & & & &  \\
\hline
Diverse programming & & & &  \\
\hline
Memorising executed cases & & & &  \\
\hline
Software error effect analysis & & & &  \\
\hline
Fully defined interface & & & &  \\
\hline
Modelling  & & & &  \\
\hline
Structured methodology & & & &  \\
\hline
\end{tabular}

\section{Software code generation}
This section discusses the usage of the approach for software code generation.
It can be skipped depending the results of \ref{main_usage}.

Which criteria for software design and implementation are covered by the methodology
(see EN50128 table A.4) :

\begin{tabular}{|l | c | c | c | c|}
\hline
& \textcolor{green}{Author} & \textcolor{blue}{Assessor 1} & \textcolor{magenta}{Assessor 2} & Total \\
\hline
Formal methods  & & & &  \\
\hline 
Modeling  & & & &  \\
\hline
Modular approach (mandatory) & & & &  \\
\hline
Components & & & &  \\
\hline
Design and coding standards (mandatory) & & & &  \\
\hline
Strongly typed programming language & & & &  \\
\hline

\end{tabular}

\section{Other comments}
Please to  give free comments on the approach (less than a page).



