\chapter{ProR}
\label{sec:pror}

\section{Author and Assessors}

\begin{description}
\item[\textcolor{green}{Author}] Michael Jastram, Formal Mind GmbH
\item[\textcolor{blue}{Assessor 1}] First assessor of the approaches: Bernd Hekele - DB-Netz
\item[\textcolor{magenta}{Assessor 2}] Second assessor of the approaches: Christophe Ponsard - for Alstom
\end{description}

\section{Presentation}

\begin{description}
\item[Name] Eclipse ProR
\item[Web site] \url{http://eclipse.org/rmf/pror}
\item[Licence] Eclipse Public License
\end{description}

\paragraph{Abstract}

ProR is an open source tool, which is part of the Eclipse Requirements Modeling Framework (RMF) \cite{RMF}.  As the underlying data format, ProR uses ReqIF, a standard for exchange of requirements with other tools.  It therefore provides interoperability with industry-strenth tools like Rational DOORS or MKS Integrity.  It supports traceability within requiremenets, and traceability solutions for artefacts outside ProR exist.

\paragraph{Publications}

\begin{itemize}

\item \cite{RMF_Mark_Book_Jastram_2013} contains a broad overview of ProR, as well as an approach to requirements structuring and model traceability.

\item \cite{jastram_forms_2012} looks at ProR as part of an Eclipse-based systems engineering environment.

\item \cite{topcase-JaGr2011} suggests the integration of Topcased with ProR, an idea that has been picked up by the Topcased community.

\item \cite{pror-benchmark} is the openETCS benchmark, which contains a longer list of literature.

\end{itemize}

For which activities are dedicaded the means or tools (give a note from 0 to  3) :

\begin{tabular}{|l | c | c | c | c|}
\hline
& \textcolor{green}{Author} & \textcolor{blue}{Assessor 1} & \textcolor{magenta}{Assessor 2} & Total \\
\hline
Data Management & 2 & 2 & 2 &  6 \\
\hline
Function Management & 2 & 2 & 2 & 6 \\
\hline
Requirement Management & 3 & 3 & 3 & 9 \\
\hline
Version Management & 2 & 1 & 1 & 4 \\
\hline
Other (give details below) & 3 & 1 & 3 & 7 \\
\hline
\end{tabular}

\begin{author_comment}
ProR clearly is a specialized tool for requirements management.  While data and functions could be managed as plain text or numeric data in ProR, this would not be very productive.

\textbf{Data and Function Management.} However, ProR is capable of ``pulling information together'' from various sources.  It would be feasible to show data and functions from Papyrus in-line in a ProR specification.  Such an integration plug-in does not currently exist, but could be created with not too much effort.

\textbf{Version Management.} ProR does not contain version management, but works well with any underlying version management tool like Subversion of git, which already works today.  However, the granularity of the versioning would be file-based, which is less than perfect.  Versioning of individual requirements would be better.  Further, higher-level versioning, like baselining, currently does not exist.

\textbf{Other.} Missing from the list is Traceability Management and Model Integration.  ProR works very well in these two areas.  Specialized traceability to Event-B exists and has been evaluated in \cite{event_b_benchmark}.  At least two third-party traceability solutions exist: ReqCycle\footnote{\url{http://www.eclipsecon.org/france2013/sessions/reqcycle-coming-some-details}} is an open source solution for Toopcased (hosted on gitHub), and Yakindu Crema\footnote{\url{http://www.guersoy.net/knowledge/crema}}, which is commercial and closed-source.
\end{author_comment}

\begin{assessor1}
For me, Tracebaility is one of the bigger points which needs extension. Also, useability is not fully achieved with the current implementation (i.e., we do not have a proper way to make a document out of the database.

But, I see also a good chance to improve the functionality of the tool in an openSource project like openETCS.

\end{assessor1}

\begin{assessor2}
I agree with the author's comments. Data and function management should probably not be managed at requirements level but requirements should be able to identify and reflect them as mentioned. I am more severe for version management as this feature is quite important for the industry so we probably need to check it in more details.
\end{assessor2}

According the results of this table, some of the following sections can be skipped.

\section{Common criteria on secondary means and tools}
\label{common}
This section discusses the common criteria of the means and tools according to the project requirements on tools and the results of T7.1.

\subsection{Project and WP2 requirements}

The objectives of this list of criteria is to check if the proposed means and tools meet the main criteria of the project: open-source approaches, usability, modularity, coverage of the objectives,...

According WP2 requirements, give a note for characteristics of the use of the tool (from 0 to 3) :

\begin{tabular}{|l | c | c | c | c|}
\hline
& \textcolor{green}{Author} & \textcolor{blue}{Assessor 1} & \textcolor{magenta}{Assessor 2} & Total \\
\hline
Open Source (D2.6-02-074) & 3 & 3 & 3 & 9 \\
\hline
Portability to operating systems (D2.6-02-075) & 3 & 3 & 3 & 9 \\
\hline
Cooperation of tools (D2.6-02-076) & 3 & 3 & 3 & 9 \\
\hline
Robustness (D2.6-02-078) & 3 & 2 & 2 & 7 \\
\hline
Modularity (D2.6-02-078.1) & 3 & 2 & 3 & 8 \\
\hline
Documentation management (D2.6-02-078.02) & 2 & 2 & 2 & 6 \\
\hline
Distributed software development (D2.6-02-078.03)  & 2 & 2 & 2 & 6 \\
\hline
Simultaneous multi-users (D2.6-02-078.04)   & 2 & 2 & 2 & 6 \\
\hline
Issue tracking (D2.6-02-078.05) & 1 & 1 & 1 & 3 \\
\hline
Differences between models (D2.6-02-078.06) & 2 & 2 & 2 & 6 \\
\hline
Version management (D2.6-02-078.07) & 2 & 1 & 1 & 4 \\
\hline
Concurrent version development (D2.6-02-078.08) & 2 & 1 & 1 & 4 \\
\hline
Model-based version control (D2.6-02-078.09) & 3 & 2 & 2 & 7 \\
\hline
Role traceability (D2.6-02-078.10) & 1 & 1 & 1 & 3 \\
\hline
Safety version traceability (D2.6-02-078.11) & 2 & 2 & 2 & 6 \\
\hline
Model traceability (D2.6-02-079) & 2 & 2 & 2 & 6 \\
\hline
Tool chain integration & 3 & 3 & 2 & 8 \\
\hline
Scalability & 3 & 3 & 3 & 9 \\
\hline
User Friendliness & 2 & 1 & 2 & 5 \\
\hline
\end{tabular}

\begin{author_comment}
Generally, the data structures of the underlying requirements model will not change, as they are based on an international standard, ReqIF 1.0.1 \cite{omg_requirements_2011}.  These data structures are powerful enough to model pretty much everything described here.  However, just the existence of the right data structures does not mean that they are used properly.  In many cases, it would be preferrable to integrate a separate tool (e.g. Mylyn for issue tracking), or to constrain the behavior of the tool programmatically.  The bottom line is that for many questions the answer is: Yes, it is possible, but not very comfortable.  For a comfortable solution, development resources are required.
\end{author_comment}

\begin{assessor1}
With ProR and its indirect interface to Doors I see a great chance for openETCS. We have the option to use existing requirements documents and standards in the project. But, on the other hand, I was not able to test this so far. The lack of some basic functions limits the useability in the SSRS team. We need a priority list on things to do.
\end{assessor1}

\begin{assessor2}
Although ProR is Open Source, it relies on closed source software for some features like importing/exporting in several (closed) formats. This is not a show stopper for using ProR as it supports ReqIF (enabling tool cooperation and most industrial own the software enabling those specific features. Documentation is still minimal however come with a tutorial and accessible within the tool. Robustness looks good but not sure about the test level as the project is still in incubation (test are present in sources but did not check in detail nor try to run them). About toolchain integration, good through model (ReqIF echange) but maybe check about support of protocols like OSLC. The project published some evidence of scalability and performance on the web: \url{http://nirmalsasidharan.wordpress.com/2011/08/03/2-fast-thrice-furious}. Support of multiple/simultaneous users is possible at EMF level but there is no mention of whether/how it is available. Same for versioning which is only available through generic support thus not aware of the requirements model structure.
\end{assessor2}

\subsection{Qualification}

This section discusses how the tool can be classified according EN50128 requirements (D2.6-02-085). Some qualification shall be mandatory  if the tool is involved to design a SIL4 software.


\begin{tabular}{|l | c | c | c | c|}
\hline
& \textcolor{green}{Author} & \textcolor{blue}{Assessor 1} & \textcolor{magenta}{Assessor 2} & Total \\
\hline
Tool manual (D.2.6-01-42.02) & 2 & 2 & 2 &  6 \\
\hline
Proof of correctness (D.2.6-01-42.03) & * & * & * & * \\
\hline
Existing industrial  usage  & 2 & 1 & 1 & 4 \\
\hline
Model verification & 1 & 1 & 1 & 3 \\
\hline
Test generation & * & * & * & * \\
\hline
Simulation, execution, debugging & * & * & * & * \\
\hline
Formal proof & * & * & * & * \\
\hline
\end{tabular}

\begin{author_comment}
The sections marked with asterisk are not applicable.

\textbf{Model verification.} There is ongoing research in supporting V\&V activities by establishing a traceability to the corresponding model \cite {HalJasLad2013}.

\end{author_comment}

\begin{assessor1}
I fully support the arguments of my co-assessor.
\end{assessor1}

\begin{assessor2}
I agree with non-applicable sections as stated by the author. Note however that requirements can be traced to generated tests and to formal proofs (property to verify). About industrial usage, some interested parties are listed in: http://www.eclipse.org/proposals/modeling.mdt.rmf but some concrete reference would be welcome.
\end{assessor2}

Which scope of qualification is expected according EN50128 (section 6.7)?

Score:
\begin{description}
\item[3] already qualified for this level
\item[2] qualification possible to this level, but some elements shall be provided
\item[0] qualification not recommended for this level
\end{description}


\begin{tabular}{|l | c | c | c | c|}
\hline
& \textcolor{green}{Author} & \textcolor{blue}{Assessor 1} & \textcolor{magenta}{Assessor 2} & Total \\
\hline
class T1 & 2 & 1 & 2 &  5 \\
\hline
class T2   & * & & & * \\
\hline
class T3  & * & & & * \\
\hline
\end{tabular}

\begin{author_comment}
I don't quite understand this section.  ProR should be classified as T1 and has never been qualified.
\end{author_comment}

\begin{assessor1}
I would not underestimate the effort to achieve a classification. I also see eclipse not as structured as needed for the classification.
\end{assessor1}

\begin{assessor2}
I agree with the author's comment: requirements management tool is T1 The strong process imposed by Eclipse should help to achieve qualification.
\end{assessor2}

\paragraph{Other elements for tool certification}


\subsection{Complementarity with primary toolchain}

The objectives of this list of criteria is to check if the proposed means and tools can be easily integrated to the primary toolchain.

\subsubsection{Language}


According to the decisions and the propositions of T7.1, how the mean and approach can be adapted to or can complete the chosen language and methods:

\begin{tabular}{|l | c | c | c | c|}
\hline
& \textcolor{green}{Author} & \textcolor{blue}{Assessor 1} & \textcolor{magenta}{Assessor 2} & Total \\
\hline
SysML  & 3 & 2 & 3 & 8 \\
\hline
Scade method & 3 & 2 & 2 & 7 \\
\hline
EFS language & 3 & 2 & 3 & 8 \\
\hline
B Method & 3 & * & 3 & 6* \\
\hline
C language & * & * & * & * \\
\hline
\end{tabular}

\paragraph{SysML}
How the means or tools can complete SysML ?

\begin{author_comment}
SysML itself provides a requirements element.  However, this is little more than a box with text in it.  We already realized a prototypical implementation, where the text in the SysML box is kept in sync with an attribute of a ProR specification.

A good SysML integration would go much further, allowing diagrams (e.g. state diagrams) to be inserted into the requirements text, or to color highlight symbols in the requirements text.
\end{author_comment}

\begin{assessor1}
I do believe we can have a nice and user-friendly interface to the languages. But I also believe there is still some way to walk. 
\end{assessor1}

\begin{assessor2}
There is a large potential of complementarity as mentioned by the author.
\end{assessor2}

\paragraph{Scade, EFS, Classical B}
How the means or tools can complete the current proposals for formal modeling language ?

\begin{author_comment}
By providing a traceability between textual requirements and model elements.
\end{author_comment}

\begin{assessor1}
I do believe we can have a nice and user-friendly interface to the languages. But I also believe there is still some way to walk. 
\end{assessor1}

\begin{assessor2}
Evidence of traceability has been shown on Event-B (RODIN) and through CREMA in a more general perspective. It is feasible but not provided out-of-the box. It will probably be more difficult to achieve for Scade at it is closed source.
\end{assessor2}

\paragraph{C language}
How the means or tools can complete or be adapted to SIL4 software in C language ?

\begin{author_comment}
This is not clear.  Conceivable are traceability to code, or incorporation of test results.  How useful this would be and how much effort would be required depends on the underlying process.
\end{author_comment}

\subsubsection{Tools and platforms}

According to the decisions and the propositions of T7.1, how the mean and approach can be integrated to or can complete the chosen tools and platforms:

\begin{tabular}{|l | c | c | c | c|}
\hline
& \textcolor{green}{Author} & \textcolor{blue}{Assessor 1} & \textcolor{magenta}{Assessor 2} & Total \\
\hline
Eclipse & 3 & 3 & 3 &  9 \\
\hline
Papyrus  & 2 & 2 & 2 & 6 \\
\hline
Scade & 2 & 2 & 2 & 6 \\
\hline
EFS tools & 2 & 2 & 2 & 6 \\
\hline
B tools & * & * & * & * \\
\hline
\end{tabular}

\begin{author_comment}
\textbf{B tools.} For Event-B, an integration already exists.  Classical B is much trickier, as has been described below.
\end{author_comment}

\begin{assessor1}
I'm mostly interested in the integration with SysML Papyrus. We should have this part on priority. Scade might be something we can cover temporarily with closed source tools since Scade as such is closed source. But we need to cover the migration path  e.g. from Reqtify to ProR.
\end{assessor1}

\begin{assessor2}
Eclipse integration is available. Papyrus, Scade (Papyrus), EFS looks feasible through EMF. B (if applied at component level) comes later in the toolchain and probably not directly integrated.
\end{assessor2}

\paragraph{Eclipse}
How the means or tools can be integrated to the Eclipse platform ?

\begin{author_comment}
ProR is Eclipse-based.
\end{author_comment}

\paragraph{Papyrus}
How the means or tools can complete  Papyrus ?

\begin{author_comment}
As Papyrus is based on EMF, it is possible to implement a traceability based on the EMF model.  This has been done in the past (Event-B).  It's fairly straight forward with textual elements, and a little more involved with graphical elements, like state diagrams.

\end{author_comment}

\paragraph{Scade, EFS, Classical B}
How the means or tools can complete the current proposals for formal modeling tools ?

\begin{author_comment}
To create a traceability, ProR would need access to the other tool's model.  If the tool writes XML, this is relatively easy.  For a textual language like Classical B thinks are trickier, as we would need a parser.  Further, in Classical B, not all elements have unique identifiers, making things even more tricky.
\end{author_comment}


\section{Means and tools for data, function and requirement management}
\label{sec:management}


This section defines the criteria for the means and tools dedicated to data, function and requirement management. These activities are shared by the work packages WP3, WP4 and the activities dedicated to  SSRS.
These means and tools shall integrate the primary toolchain to  complete its gap and facilitate the integration of different activities. First of all, they  allow the management of a common repository of data, functions and requirements, shared between the models (from SSRS informal specification to code) and the verification and validation activities.
Then, they shall support traceability of requirements between models and activities, and facilitate the verification of the traceability.
Besides they shall support the design of SIL4 software with model comparison or document production facilities, and version management.

\subsection{Management activities}

Which activities, linked to help the management of SSRS definition and whole process are covered by the mean or tool  ?

\begin{tabular}{|l | c | c | c | c|}
\hline
& \textcolor{green}{Author} & \textcolor{blue}{Assessor 1} & \textcolor{magenta}{Assessor 2} & Total \\
\hline
Requirement capturing & 3 & 3 & 3 & 9 \\
\hline
Requirement management  & 3 & 3 & 3 & 9 \\
\hline
Data management & 1 & 1 & 1 & 3 \\
\hline
Function management & 1 & 1 & 1 & 3 \\
\hline
Requirement traceability  & 3 & 3 & 3 & 9 \\
\hline
Model traceability & 2 & 2 & 2 & 6 \\
\hline
Function architecture & 1 & 1 & 1 & 3 \\
\hline
Version management & 2 & 2 & 2 & 6 \\
\hline
Model comparison & 2 & 2 & 2 & 6 \\
\hline
Documentation production & 2 & 2 & 2 & 6 \\
\hline
Others (give details) & n/a & & n/a & n/a \\
\hline
\end{tabular}

\subsection{Input Artifacts}

Which artifacts are used as input of the mean or tool  ?


\begin{tabular}{|l | c | c | c | c|}
\hline
& \textcolor{green}{Author} & \textcolor{blue}{Assessor 1} & \textcolor{magenta}{Assessor 2} & Total \\
\hline
Informal description & 3 & 3 & 3 & 9 \\
\hline
Structured description & 3 & 3 & 3 & 9 \\
\hline
Spread sheet & 3 & 1 & 2 & 6 \\
\hline
XML files & 3 & 3 & 2 & 8 \\
\hline
EFS model & 1 & 1 & 1 & 3 \\
\hline
DSL & 2 & 2 & 2 & 6 \\
\hline
Others (give details) & n/a & & & n/a \\
\hline
\end{tabular}

\begin{author_comment}

\textbf{Informal/Structured description.} The only input format ProR currently accepts is ReqIF.  However, ReqIF can be created with many tools, including Rational DOORS.  At least for the time being, this could be used as a universal converter, allowing Word, Spreadsheets, plain text, etc. to be converted to ReqIF.

\textbf{EFS model.} It would make little sense to ``convert'' EFS to ReqIF.  Instead, it would make sense to create an adapter that would allow traceability to/from EFS.

\textbf{DSL.} There have been prototypical implementations of XText (DSL framework) and ProR.  Such an implementation allows to edit in the ProR-cells with a DSL editor.
\end{author_comment}

\begin{assessor1}
I would like to see more flexibility in the interface to spreadsheets. Import of data from spreadsheets would allow for more efficiency in daily work. 
\end{assessor1}

\begin{assessor2}
Format other than ReqIF require an external tool. The way to manage import/export (possibly partial) is not well documented.
\end{assessor2}

\subsection{Output Artifacts}

Which artifacts are used as output of the mean or tool  ?


\begin{tabular}{|l | c | c | c | c|}
\hline
& \textcolor{green}{Author} & \textcolor{blue}{Assessor 1} & \textcolor{magenta}{Assessor 2} & Total \\
\hline
Informal description & 2 & 1 & 2 & 5 \\
\hline
Structured description & 3 & 3 & 2 & 8 \\
\hline
Spread sheet & 2 & 1 & 2 & 5 \\
\hline
XML files & 3 & 3 & 3 & 9 \\
\hline
EFS model & 0 & 0 & 0 & 0 \\
\hline
DSL & 2 & 2 & 2 & 6 \\
\hline
Others (give details) & & & & \\
\hline
\end{tabular}

\begin{assessor1}
We need some means to transport the collected requirements into a professional document. Example: printout of requirements for review with partners not being familiar with ProR. The interface to spreadsheets would allow for easy enhancement of quite sophisticated functionality (e.g., filtering, pivots).
\end{assessor1}

\begin{assessor2}
Export require external converter. There does not seem to be a report generator nor direct way to query the model at UI level (e.g. using an OCL query) although this is probably easy to achieve using the ReqIF API or EMF. The extension point provided may also help.
\end{assessor2}

\subsection{Requirement Management}

This section is link to requirement definition and management activities.

Are these criteria covered by the tool or mean ?


\begin{tabular}{|l | c | c | c | c|}
\hline
& \textcolor{green}{Author} & \textcolor{blue}{Assessor 1} & \textcolor{magenta}{Assessor 2} & Total \\
\hline
Editing of Textual Requirements & 3 & 3 & 3 & 9 \\
\hline
Represent Relations between Req (Textual-Based) & 3 & 3 & 3 & 9 \\
\hline
Represent Relations between Req (Modelling-Based) & 2 & 2 & 2 & 6 \\
\hline
Glossary and Abbreviation handling (Linked to Req)) & 2 & 2 & 2 & 6 \\
\hline
Traceability of Textual Requirements to Modelling & 2 & 2 & 2 & 6 \\
\hline
Import/Export of Industrial Standard Data (e.g., REQIF) & 3 & 3 & 3 & 9 \\
\hline
Documentation generation & 2 & 1 & 1 & 4 \\
\hline
Search and Filtering functions & 1 & 1 & 1 & 3 \\
\hline
Others (give details) & & & & \\
\hline
\end{tabular}

\begin{assessor1}
We need to see how we can get an easy and user-friendly workaround for those missing functions. 
\end{assessor1}

\begin{assessor2}
There does not seem to be a report generator nor direct way to query the model at UI level (e.g. using an OCL query) although this is probably easy to achieve using the ReqIF API or EMF. The extension point provided may also help.
\end{assessor2}


\section{Other comments}



\begin{comment}
This section is available for the author or the assessors to  complete the description and criteria.
\end{comment}



